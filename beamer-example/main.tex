\documentclass[
    10pt,
    aspectratio=1610,
    xcolor={dvipsnames,pst},
    % handout
]{beamer}
\usetheme[
    subsectionpage=progressbar,
    progressbar=frametitle,
    block=transparent
]{moloch}
\mode<presentation>
%%%%%%%%%%%%%%%%%%%%%%%%%%%%%%%%%%%%%%%%%%%%%%%%%%%%%%%%%%%%%%%%%%%%%%%%%%%%%%%%
%%%%%%%%%%%%% personal modification of the moloch/metropolis theme %%%%%%%%%%%%%
%%%%%%%%%%%%%%%%%%%%%%%%%%%%%%%%%%%%%%%%%%%%%%%%%%%%%%%%%%%%%%%%%%%%%%%%%%%%%%%%
% frametitle with right-aligned "section, subsection"
\newenvironment{Frame}[1]{
    \begin{frame}{#1 \hspace{0pt plus 1filll} \scriptsize \(\triangleleft\)\;\subsecname\;\(\triangleleft\)\;\secname}\vspace*{\fill}
}{\vspace*{\fill}\end{frame}}

% inner theme
\useinnertheme{rectangles}

% color palette
\definecolor{um-blue}{HTML}{002855}
\definecolor{um-gold}{HTML}{84754E}
\definecolor{um-red}{HTML}{Ef3340}
\definecolor{um-yellow}{HTML}{84754E}
% color theme
\colorlet{color-frametitle}{um-blue}
\colorlet{color-progressbar}{um-gold}

\setbeamercolor{frametitle}{fg=black!2, bg=color-frametitle}
\setbeamercolor{progress bar}{fg=color-progressbar, bg=color-progressbar!20}
\setbeamercolor{item projected}{fg=black!2, bg=color-progressbar}
\setbeamercolor{itemize item}{fg=color-progressbar}

% make progress bar's width larger
\makeatletter
\setlength{\moloch@titleseparator@linewidth}{2.5pt}
\setlength{\moloch@progressonsectionpage@linewidth}{2.5pt}
\setlength{\moloch@progressinheadfoot@linewidth}{2.5pt}
\makeatother

% make footnotesize smaller
\let\oldfootnotesize\footnotesize
\renewcommand*{\footnotesize}{\oldfootnotesize\tiny}

% footnote without counter
\newcommand\blfootnote[1]{%
  \begingroup
  \renewcommand\thefootnote{}\footnote{#1}%
  \addtocounter{footnote}{-1}%
  \endgroup
}

% footnotetext without counter
\newcommand\blfootnotetext[1]{%
  \begingroup
  \renewcommand\thefootnotetext{}\footnotetext{#1}%
  \addtocounter{footnotetext}{-1}%
  \endgroup
}

%%%%%%%%%%%%%%%%%%%%%%%%%%%%%%%%%%%%%%%%%%%%%%%%%%%%%%%%%%%%%%%%%%%%%%%%%%%%%%%%
%%%%%%%%%%%%%%%%%%%%%%%%%%%%%%%%%%%% fonts %%%%%%%%%%%%%%%%%%%%%%%%%%%%%%%%%%%%%
%%%%%%%%%%%%%%%%%%%%%%%%%%%%%%%%%%%%%%%%%%%%%%%%%%%%%%%%%%%%%%%%%%%%%%%%%%%%%%%%
\usepackage{xeCJK}
\setCJKmainfont{思源宋体}
\setCJKsansfont{思源黑体}
\setCJKmonofont{思源等宽}

%%%%%%%%%%%%%%%%%%%%%%%%%%%%%%%%%%%%%%%%%%%%%%%%%%%%%%%%%%%%%%%%%%%%%%%%%%%%%%%%
%%%%%%%%%%%%%%%%%%%%%%%%%%%%%%%%%%%% color %%%%%%%%%%%%%%%%%%%%%%%%%%%%%%%%%%%%%
%%%%%%%%%%%%%%%%%%%%%%%%%%%%%%%%%%%%%%%%%%%%%%%%%%%%%%%%%%%%%%%%%%%%%%%%%%%%%%%%
\definecolorseries{marknode-color-series}{hsb}{last}[hsb]{0.0,0.12,0.95}[hsb]{0.95,0.12,0.95}
\definecolorseries{annotation-color-series}{hsb}{last}[hsb]{0.0,0.8,0.65}[hsb]{0.95,0.8,0.65}
\resetcolorseries[8]{marknode-color-series}
\resetcolorseries[8]{annotation-color-series}

%%%%%%%%%%%%%%%%%%%%%%%%%%%%%%%%%%%%%%%%%%%%%%%%%%%%%%%%%%%%%%%%%%%%%%%%%%%%%%%%
%%%%%%%%%%%%%%%%%%%%%%%%%%%%%%%%%%%% glyph %%%%%%%%%%%%%%%%%%%%%%%%%%%%%%%%%%%%%
%%%%%%%%%%%%%%%%%%%%%%%%%%%%%%%%%%%%%%%%%%%%%%%%%%%%%%%%%%%%%%%%%%%%%%%%%%%%%%%%
\usepackage{fontawesome5}
\usepackage{bbding}
\usepackage{pifont}
\newcommand{\cmark}{\ding{51}}
\newcommand{\xmark}{\ding{55}}
\usepackage{romannum}
\usepackage{stmaryrd} % for \mapsfrom (inverse of \mapsto)

%%%%%%%%%%%%%%%%%%%%%%%%%%%%%%%%%%%%%%%%%%%%%%%%%%%%%%%%%%%%%%%%%%%%%%%%%%%%%%%%
%%%%%%%%%%%%%%%%%%%%%%%%%%%%%%%%%%%% table %%%%%%%%%%%%%%%%%%%%%%%%%%%%%%%%%%%%%
%%%%%%%%%%%%%%%%%%%%%%%%%%%%%%%%%%%%%%%%%%%%%%%%%%%%%%%%%%%%%%%%%%%%%%%%%%%%%%%%
\usepackage{booktabs}

% ref: https://tex.stackexchange.com/a/614273/240783
\usepackage{tabularray}
\UseTblrLibrary{booktabs}

%%%%%%%%%%%%%%%%%%%%%%%%%%%%%%%%%%%%%%%%%%%%%%%%%%%%%%%%%%%%%%%%%%%%%%%%%%%%%%%%
%%%%%%%%%%%%%%%%%%%%%%%%%%%%%%%%%%% caption %%%%%%%%%%%%%%%%%%%%%%%%%%%%%%%%%%%%
%%%%%%%%%%%%%%%%%%%%%%%%%%%%%%%%%%%%%%%%%%%%%%%%%%%%%%%%%%%%%%%%%%%%%%%%%%%%%%%%
\usepackage{caption}
\captionsetup{font={scriptsize},
              labelfont={scriptsize},
              textfont={scriptsize},
              % hypcap=false,
              % format=hang,
              % margin=1cm
             }

%%%%%%%%%%%%%%%%%%%%%%%%%%%%%%%%%%%%%%%%%%%%%%%%%%%%%%%%%%%%%%%%%%%%%%%%%%%%%%%%
%%%%%%%%%%%%%%%%%%%%%%%%%%%%%%%%%%% graphics %%%%%%%%%%%%%%%%%%%%%%%%%%%%%%%%%%%
%%%%%%%%%%%%%%%%%%%%%%%%%%%%%%%%%%%%%%%%%%%%%%%%%%%%%%%%%%%%%%%%%%%%%%%%%%%%%%%%
% \usepackage{graphicx} % pre-loaded by beamer
\graphicspath{{./images}}

\usepackage{forest} % mindmap

% \usetikzlibrary{calc,tikzmark,arrows.meta,fit,positioning,decorations.markings}

%%%%%%%%%%%%%%%%%%%%%%%%%%%%%%%%%%%%%%%%%%%%%%%%%%%%%%%%%%%%%%%%%%%%%%%%%%%%%%%%
%%%%%%%%%%%%%%%%%%%%%%%%% equation annotation by TikZ %%%%%%%%%%%%%%%%%%%%%%%%%%
%%%%%%%%%%%%%%%%%%%%%%%%%%%%%%%%%%%%%%%%%%%%%%%%%%%%%%%%%%%%%%%%%%%%%%%%%%%%%%%%
% \usepackage{tikz}
% \usepackage[dvipsnames]{xcolor}
\usetikzlibrary{calc,tikzmark}
% \usepackage{tcolorbox}
\usepackage{makecell}
\usepackage{xstring}
%%%%%%%%%%%%%%%%%%%%%%%%%%%%%%%%%%%%%%%%%%%%%%%%%%%%%%%%%%%%%%%%%%%%%%%%%%%%%%%%
% Usage:
% \annotatedEquation{1: color/colorseries}{2: node name}{3: node direction}{4: x shift}{5: y shift}{6: anchor direction}{7: color/colorseries name}{8: annotation}{9: baseline direction}
%%%%%%%%%%%%%%%%%%%%%%%%%%%%%%%%%%%%%%%%%%%%%%%%%%%%%%%%%%%%%%%%%%%%%%%%%%%%%%%%
\newenvironment{annotatedEquationEnv}{
    \begin{tikzpicture}[
        overlay, remember picture,
        >=stealth, <-,
        nodes={align=left,inner ysep=1pt},
    ]
}
{
    \end{tikzpicture}
}
\newcommand*{\annotatedEquation}[9]{%
     \IfEqCase{#1}{%
        {color}{%
            \path (#2.#3) ++ (#4,#5) node [anchor=#6] (#2-annotate) { \color{#7} \scriptsize \makecell[l]{#8} };%
            \draw [#7] (#2.#3) |- (#2-annotate.south #9);%
        }%
        {colorseries}{%
            \path (#2.#3) ++ (#4,#5) node [anchor=#6] (#2-annotate) { \color{#7!!} \scriptsize \makecell[l]{#8} };%
            \draw [#7!!] (#2.#3) |- (#2-annotate.south #9);%
            \textcolor{#7!!+}{}%
        }%
    }[\PackageError{annotatedEquation}{Undefined option to annotatedEquation #1}{}]%
}%

%%%%%%%%%%%%%%%%%%%%%%%%%%%%%%%%%%%%%%%%%%%%%%%%%%%%%%%%%%%%%%%%%%%%%%%%%%%%%%%%
%%%%%%%%%%%%%%%%%%%%%%%%%% figure annotation by TikZ  %%%%%%%%%%%%%%%%%%%%%%%%%%
%%%%%%%%%%%%%%%%%%%%%%%%%%%%%%%%%%%%%%%%%%%%%%%%%%%%%%%%%%%%%%%%%%%%%%%%%%%%%%%%
% \usepackage{tikz}
%%%%%%%%%%%%%%%%%%%%%%%%%%%%%%%%%%%%%%%%%%%%%%%%%%%%%%%%%%%%%%%%%%%%%%%%%%%%%%%%
% Usage:
% \begin{annotatedFigureEnv}
%     {\includegraphics[width=0.5\linewidth]{example-image}}
%     \annotatedFigureBox{bottom-left}{top-right}{label}{label-position}
% \end{annotatedFigureEnv}
% Usage:
% \annotatedFigure{bottom-left}{top-right}{label}{label-position}
% Usage:
% \annotatedFigureImpl{1: bottom-left}{2: top-right}{3: label}{4: label-position}{5: box-color}{6: label-color}{7: border-color}{8: text-color}
% Usage:
% \figureBox{bottom-left}{top-right}{color}{thickness}
%%%%%%%%%%%%%%%%%%%%%%%%%%%%%%%%%%%%%%%%%%%%%%%%%%%%%%%%%%%%%%%%%%%%%%%%%%%%%%%%
\newcommand*\annotatedFigureImpl[8]{
    \draw [#5, ultra thick] (#1) rectangle (#2); 
    \node at (#4) [fill=#6, thick, shape=rectangle, draw=#7, inner sep=2.5pt, font=\small\sffamily, text=#8] { #3 };
}
\newcommand*\annotatedFigure[4]{
    \annotatedFigureImpl{#1}{#2}{#3}{#4}{color-progressbar}{color-progressbar}{color-progressbar}{black!2}
}
\newcommand*\annotatedFigureText[4]{
    \node[draw=none, anchor=south west, text=#2, inner sep=0, text width=#3\linewidth,font=\sffamily] at (#1) {#4};
}
\newenvironment{annotatedFigureEnv}[1]{
    \centering
    \begin{tikzpicture}
        \node[anchor=south west,inner sep=0] (image) at (0,0) { #1 };
        \begin{scope}[x={(image.south east)},y={(image.north west)}]
}
{
        \end{scope}
    \end{tikzpicture}
}
\newcommand*\figureBox[4]{\draw[#3,#4,rounded corners] (#1) rectangle (#2);}

%%%%%%%%%%%%%%%%%%%%%%%%%%%%%%%%%%%%%%%%%%%%%%%%%%%%%%%%%%%%%%%%%%%%%%%%%%%%%%%%
%%%%%%%%%%%%%%%%%%%%%%%%%%%%%%% Taxonomy by TikZ %%%%%%%%%%%%%%%%%%%%%%%%%%%%%%%
%%%%%%%%%%%%%%%%%%%%%%%%%%%%%%%%%%%%%%%%%%%%%%%%%%%%%%%%%%%%%%%%%%%%%%%%%%%%%%%%
% ref: https://tex.stackexchange.com/a/112471/240783
% ref: https://tex.stackexchange.com/a/357412/240783
%%%%%%%%%%%%%%%%%%%%%%%%%%%%%%%%%%%%%%%%%%%%%%%%%%%%%%%%%%%%%%%%%%%%%%%%%%%%%%%%
\usetikzlibrary{shadows}
\makeatletter
\def\tikzopacityregister{.2} % the opacity of the shadows
\tikzset{
  opacity/.append code={
    \pgfmathsetmacro\tikzopacityregister{#1*\tikzopacityregister}
  },
  opacity aux/.code={ % this is the original definition of opacity
    \tikz@addoption{\pgfsetstrokeopacity{#1}\pgfsetfillopacity{#1}}
  },
  every shadow/.style={opacity aux=\tikzopacityregister}
}
\makeatother
\tikzset{
    my node for tree/.style={
            text=black,
            draw=lightgray!50,
            fill=lightgray!20,
            % inner color=lightgray!5,
            % outer color=lightgray!20,
            thick,
            minimum width=12mm,
            minimum height=6mm,
            rounded corners=3,
            text height=1.5ex,
            text depth=0ex,
            font={\sffamily},
            drop shadow,
        },
    invisible/.style={opacity=0,text opacity=0},
    visible on/.style={alt=#1{}{invisible}},
    alt/.code args={<#1>#2#3}{%
      \alt<#1>{\pgfkeysalso{#2}}{\pgfkeysalso{#3}} % \pgfkeysalso doesn't change the path
    },
}
\forestset{
    visible on/.style={
        for tree={
            /tikz/visible on={#1},
            edge+={/tikz/visible on={#1}},
        }
    },
    my tree/.style={
            my node for tree,
            s sep+=4pt,
            l sep+=10pt,
            grow'=east,
            edge+={lightgray},
            parent anchor=east,
            child anchor=west,
            edge path={
                    \noexpand\path [draw, \forestoption{edge}] (!u.parent anchor) -- +(10pt,0) |- (.child anchor)\forestoption{edge label};
                },
            if={isodd(n_children())}{
                    for children={
                            if={equal(n,(n_children("!u")+1)/2)}{calign with current}{}
                        }
                }{},
        }
}

%%%%%%%%%%%%%%%%%%%%%%%%%%%%%%%%%%%%%%%%%%%%%%%%%%%%%%%%%%%%%%%%%%%%%%%%%%%%%%%%
%%%%%%%%%%%%%%%%%%%%%%%%%%%%% pdf, svg, animation %%%%%%%%%%%%%%%%%%%%%%%%%%%%%%
%%%%%%%%%%%%%%%%%%%%%%%%%%%%%%%%%%%%%%%%%%%%%%%%%%%%%%%%%%%%%%%%%%%%%%%%%%%%%%%%
\usepackage{pdfpages}
\usepackage{svg}
\usepackage{animate}

%%%%%%%%%%%%%%%%%%%%%%%%%%%%%%%%%%%%%%%%%%%%%%%%%%%%%%%%%%%%%%%%%%%%%%%%%%%%%%%%
%%%%%%%%%%%%%%%%%%%%%%%%%%%%%%%%%%%% others %%%%%%%%%%%%%%%%%%%%%%%%%%%%%%%%%%%%
%%%%%%%%%%%%%%%%%%%%%%%%%%%%%%%%%%%%%%%%%%%%%%%%%%%%%%%%%%%%%%%%%%%%%%%%%%%%%%%%
\usepackage{lipsum}

\usepackage{adjustbox}

% Usage: Map numbers to alphabets
% Example: \Letter{1} prints A, \Letter{2} prints B
\makeatletter
\newcommand{\Letter}[1]{\@Alph{#1}}
\makeatother

%%%%%%%%%%%%%%%%%%%%%%%%%%%%%%%%%%%%%%%%%%%%%%%%%%%%%%%%%%%%%%%%%%%%%%%%%%%%%%%%
%%%%%%%%%%%%%%%%%%%%%%%%%%%%%%%%%% reference %%%%%%%%%%%%%%%%%%%%%%%%%%%%%%%%%%%
%%%%%%%%%%%%%%%%%%%%%%%%%%%%%%%%%%%%%%%%%%%%%%%%%%%%%%%%%%%%%%%%%%%%%%%%%%%%%%%%
\usepackage[backref=true, natbib=true, backend=biber, style=authoryear-icomp, useprefix=true, style=ieee]{biblatex}
\AtBeginBibliography{\scriptsize}
\addbibresource{sample.bib}

\usepackage{hyperref}
\hypersetup{
    colorlinks=true,
    linkcolor=.,
    anchorcolor=.,
    filecolor=.,
    menucolor=.,
    runcolor=.,
    urlcolor=cyan,
    citecolor=cyan,
}

%%%%%%%%%%%%%%%%%%%%%%%%%%%%%%%%%%%%%%%%%%%%%%%%%%%%%%%%%%%%%%%%%%%%%%%%%%%%%%%%
%%%%%%%%%%%%%%%%%%%%%%%%%%%%%%%%%% meta info %%%%%%%%%%%%%%%%%%%%%%%%%%%%%%%%%%%
%%%%%%%%%%%%%%%%%%%%%%%%%%%%%%%%%%%%%%%%%%%%%%%%%%%%%%%%%%%%%%%%%%%%%%%%%%%%%%%%
\title{Academic Slides}
\subtitle{a template based on Beamer, TikZ, ...}
\author{Shuqi XIAO}
% \logo{
%     \includegraphics[width=1.5cm]{example-image}
% }
% \titlegraphic{
%     \begin{tikzpicture}[remember picture, overlay]
%         \usetikzlibrary{calc}
%         \node [anchor=north east] at ($(current page.north east)+(-2.5em,-2.5em)$) {
%             \includegraphics[width=3cm]{example-image}
%         };
%     \end{tikzpicture}
% }

%%%%%%%%%%%%%%%%%%%%%%%%%%%%%%%%%%%%%%%%%%%%%%%%%%%%%%%%%%%%%%%%%%%%%%%%%%%%%%%%
%%%%%%%%%%%%%%%%%%%%%%%%%%%%%%%%%%% document %%%%%%%%%%%%%%%%%%%%%%%%%%%%%%%%%%%
%%%%%%%%%%%%%%%%%%%%%%%%%%%%%%%%%%%%%%%%%%%%%%%%%%%%%%%%%%%%%%%%%%%%%%%%%%%%%%%%
\begin{document}

\maketitle

\begin{frame}{Outline}
	\tableofcontents
\end{frame}

\section{Basic Examples}

\subsection{Text}
\begin{Frame}{中文}
	\par 测试
	\begin{itemize}
		\setlength{\itemsep}{1.5ex}
		\item 蔽芾甘棠,\textbf{勿}翦{\large 勿}伐,{\scriptsize 召}伯所{\huge 茇}。
	\end{itemize}
\end{Frame}
\subsection{Color}
\begin{Frame}{Color}
	\begin{figure}[htbp]
		\centering
		\includegraphics[width=0.7\linewidth]{OLxcolorList2.png}
		\vspace*{0.5ex}
		\caption{Colors supported by \texttt{xcolor} package with \texttt{dvipsnames} option}
	\end{figure}
	\blfootnote{\href{https://es.overleaf.com/learn/latex/Using_colors_in_LaTeX}{A tutorial from overleaf: using colors in \LaTeX}}
\end{Frame}

\subsection{Figure}
\begin{Frame}{Figure}
	Look at figure~\ref{fig:example-image-a} and \ref{fig:example-image-b}.
	\vspace*{1em}
	\begin{figure}[htbp]
		\begin{minipage}[c]{0.45\linewidth}
			\centering
			\includegraphics[width=0.7\linewidth]{example-image-a}
			\vspace*{0.5ex}
			\caption{Image A}
			\label{fig:example-image-a}
		\end{minipage}
		\hspace{\fill}
		\begin{minipage}[c]{0.45\linewidth}
			\centering
			\includegraphics[width=0.7\linewidth]{example-image-b}
			\vspace*{0.5ex}
			\caption{Image B}
			\label{fig:example-image-b}
		\end{minipage}
		\vspace*{0.5ex}
		\caption*{Example Images}
		\label{fig:example-images}
	\end{figure}
\end{Frame}

\subsection{Table}
\begin{Frame}{Table}
	\begin{table}[htbp]
		\centering
		\scriptsize
		\begin{tblr}{p{2cm}|XXXX|X|X|X}
			\toprule
			Methods       & LPIPS \(\downarrow\) & SSIM \(\uparrow\) & PSNR \(\mathrm{dB}\uparrow\) & {Depth                          \\L1 \\ \(\mathrm{cm}\downarrow\)} & ATE RMSE \(\mathrm{cm}\downarrow\) & {mIoU         \\ \(\%\uparrow\)} & {FPS\\ \(\mathrm{Hz}\uparrow\)} \\
			\midrule
			NIDS-SLAM     & {0.011}              & {0.980}           & {35.76}                      & 0.56   & 0.80   & 82.37   & -   \\
			DNS-SLAM      & 0.119                & 0.963             & 22.96                        & 3.16   & 0.45   & 84.77   & -   \\
			SNI-SLAM      & 0.235                & 0.935             & 29.43                        & 0.77   & 0.46   & 87.41   & -   \\
			\midrule[dashed]
			MonoGS        & {0.068}              & 0.954             & {34.83}                      & -      & 0.58   & N/A     & 1.7 \\
			SplaTAM       & 0.10                 & 0.97              & 34.11                        & {0.49} & {0.36} & N/A     & 1.1 \\
			\midrule[dashed]
			NEDS-SLAM     & 0.088                & 0.962             & 34.76                        & {0.47} & {0.35} & {90.78} & -   \\
			SemGauss-SLAM & {0.062}              & {0.982}           & {35.03}                      & 0.50   & {0.33} & {94.22} & -   \\
			SGS-SLAM      & 1.096                & {0.973}           & 34.15                        & {0.36} & 0.41   & {92.72} & -   \\
			\textbf{Ours} & 0.086                & {0.975}           & 34.73                        & {-}    & 0.67   & {91.14} & 0.8 \\
			\bottomrule
		\end{tblr}
		\caption{Comparison of average performance on Replica dataset}
		% \label{table:}
	\end{table}
\end{Frame}

\subsection{Animation}

\begin{Frame}{Text}
	TODO
\end{Frame}

\begin{Frame}{Movie}
	TODO
\end{Frame}

\section{Tikz Examples}

\subsection{Figure Annotation}
\begin{Frame}{Figure Annotation}
	\begin{figure}[htbp]
		\centering
		\begin{annotatedFigureEnv}
			{\includegraphics[width=0.4\linewidth]{example-image}}
			\annotatedFigure{0.1,0.2}{0.3,0.4}{1}{0.1,0.2}
			\annotatedFigure{0.6,0.6}{0.9,0.9}{2}{0.6,0.6}
		\end{annotatedFigureEnv}
	\end{figure}
	\vspace*{\fill}
	\begin{enumerate}
		\setlength{\itemsep}{1.5ex}
		\item Hello
		\item Hi
	\end{enumerate}
\end{Frame}

\subsection{Equation Annotation}

\begin{Frame}{Equation Annotation \romannum{1}}
	\vspace*{-10em}
	\begin{block}{Recommendations on Color Palette}
		\vspace*{1.5ex}
		\begin{itemize}
			\setlength{\itemsep}{1.5ex}
			\item Marknode: super-low saturation \& super-high brightness.
			\item Annotation: high saturation \& low brightness.
		\end{itemize}
	\end{block}
	\vspace*{-2em}
	\begin{figure}[htbp]
		\centering
		\begin{equation}
			\tikzmarknode{n0}{\colorbox{marknode-color-series!![0]}{\(\Theta\)}}
			\tikzmarknode{n1}{\colorbox{marknode-color-series!![1]}{\(_{k}\)}}
			=
			\left\{
			\tikzmarknode{n2}{\colorbox{marknode-color-series!![2]}{\(\mathbf{x}\)}}_k,
			\tikzmarknode{n3}{\colorbox{marknode-color-series!![3]}{\(\mathbf{q}\)}}_k,
			\tikzmarknode{n4}{\colorbox{marknode-color-series!![4]}{\(\mathbf{s}\)}}_k,
			\tikzmarknode{n5}{\colorbox{marknode-color-series!![5]}{\(\alpha\)}}_k,
			\tikzmarknode{n6}{\colorbox{marknode-color-series!![6]}{\(\mathbf{c}\)}}_k,
			\tikzmarknode{n7}{\colorbox{marknode-color-series!![7]}{\(\mathbf{f}\)}}_k
			\right\}
		\end{equation}
		\begin{annotatedEquationEnv}
			\annotatedEquation{colorseries}{n0}{south}{0}{-0.5em}{north east}{annotation-color-series}{optimizable attributes}{west}
			\annotatedEquation{colorseries}{n1}{south}{0}{-2.5em}{north east}{annotation-color-series}{\(\in \left\{0,1,\cdots,N\right\}\), \\[1ex]index of Gaussians}{west}
			\annotatedEquation{colorseries}{n2}{south}{0}{-8em}{north west}{annotation-color-series}{\(\in \mathbb{R}^3\), position}{east}
			\annotatedEquation{colorseries}{n3}{south}{0}{-6.5em}{north west}{annotation-color-series}{\(\in \mathrm{SO}(3)\), orientation}{east}
			\annotatedEquation{colorseries}{n4}{south}{0}{-5em}{north west}{annotation-color-series}{\(\in \mathbb{R}^{3}\), scale}{east}
			\annotatedEquation{colorseries}{n5}{south}{0}{-3.5em}{north west}{annotation-color-series}{\(\in [0,1]\), opacity}{east}
			\annotatedEquation{colorseries}{n6}{south}{0}{-2em}{north west}{annotation-color-series}{\(\in \mathbb{R}^{3n}\), color}{east}
			\annotatedEquation{colorseries}{n7}{south}{0}{-0.5em}{north west}{annotation-color-series}{\(\in \mathbb{R}^{D}\), semantic embedding}{east}
		\end{annotatedEquationEnv}
	\end{figure}
\end{Frame}

\begin{Frame}{Equation Annotation \romannum{2}}
	\vspace*{-3em}
	A counter-example,
	\vspace*{-2em}
	\begin{figure}[htbp]
		\centering
		\begin{equation}
			\tikzmarknode{n0}{\colorbox{annotation-color-series!![0]}{\(\Theta\)}}
			\tikzmarknode{n1}{\colorbox{annotation-color-series!![1]}{\(_{k}\)}}
			=
			\left\{
			\tikzmarknode{n2}{\colorbox{annotation-color-series!![2]}{\(\mathbf{x}\)}}_k,
			\tikzmarknode{n3}{\colorbox{annotation-color-series!![3]}{\(\mathbf{q}\)}}_k,
			\tikzmarknode{n4}{\colorbox{annotation-color-series!![4]}{\(\mathbf{s}\)}}_k,
			\tikzmarknode{n5}{\colorbox{annotation-color-series!![5]}{\(\alpha\)}}_k,
			\tikzmarknode{n6}{\colorbox{annotation-color-series!![6]}{\(\mathbf{c}\)}}_k,
			\tikzmarknode{n7}{\colorbox{annotation-color-series!![7]}{\(\mathbf{f}\)}}_k
			\right\}
		\end{equation}
		\begin{annotatedEquationEnv}
			\annotatedEquation{colorseries}{n0}{south}{0}{-0.5em}{north east}{marknode-color-series}{optimizable attributes}{west}
			\annotatedEquation{colorseries}{n1}{south}{0}{-2.5em}{north east}{marknode-color-series}{\(\in \left\{0,1,\cdots,N\right\}\), \\[1ex]index of Gaussians}{west}
			\annotatedEquation{colorseries}{n2}{south}{0}{-8em}{north west}{marknode-color-series}{\(\in \mathbb{R}^3\), position}{east}
			\annotatedEquation{colorseries}{n3}{south}{0}{-6.5em}{north west}{marknode-color-series}{\(\in \mathrm{SO}(3)\), orientation}{east}
			\annotatedEquation{colorseries}{n4}{south}{0}{-5em}{north west}{marknode-color-series}{\(\in \mathbb{R}^{3}\), scale}{east}
			\annotatedEquation{colorseries}{n5}{south}{0}{-3.5em}{north west}{marknode-color-series}{\(\in [0,1]\), opacity}{east}
			\annotatedEquation{colorseries}{n6}{south}{0}{-2em}{north west}{marknode-color-series}{\(\in \mathbb{R}^{3n}\), color}{east}
			\annotatedEquation{colorseries}{n7}{south}{0}{-0.5em}{north west}{marknode-color-series}{\(\in \mathbb{R}^{D}\), semantic embedding}{east}
		\end{annotatedEquationEnv}
	\end{figure}
\end{Frame}

\subsection{Mindmap}
\begin{Frame}{Mindmap (Concept)}
	\centering
	\definecolor{0_color_mindmap}{HTML}{F6A4E2}
	\definecolor{1_color_mindmap}{HTML}{F6E1A4}
	\definecolor{2_color_mindmap}{HTML}{A4F6B8}
	\definecolor{3_color_mindmap}{HTML}{A4B9F6}
	\par \resizebox{!}{0.6\textheight}{
		\begin{tikzpicture}
			\usetikzlibrary{mindmap}
			\path[
				mindmap,
				grow cyclic,
				every node/.style={concept},
				text=black,
				root concept/.append style={
						concept color=color-frametitle,
						font=\Huge\bfseries,
						text width=15em,
						text=white,
					},
				level 1 concept/.append style={
						font=\Huge\bfseries,
						text width=20em,
						% sibling angle=360/4,
						% rotate=45,
						level distance=25em,
					},
				accuracy/.append style={concept color=0_color_mindmap},
				efficiency/.append style={concept color=1_color_mindmap},
				consistency/.append style={concept color=2_color_mindmap},
				interactivity/.append style={concept color=3_color_mindmap},
				level 2 concept/.append style={
						font=\Large\bfseries,
						level distance=15em,
					},
				level 3 concept/.append style={
						font=\large\bfseries,
					},
			]
			node[root concept] {Semantic\\[1ex]NeRF/3DGS} [clockwise from=-90] {
				child[accuracy] {
						node[concept] {Accuracy} [counterclockwise from=180] {
								% child {node[concept] {1}}
								% child {node[concept] {2}}
								% child {node[concept] {3}}
								% child {node[concept] {4}}
							}
					}
				child[efficiency] {
						node[concept] {Efficiency} [counterclockwise from=90] {
								% child {node[concept] {1}}
								% child {node[concept] {2}}
								% child {node[concept] {3}}
								% child {node[concept] {4}}
							}
					}
				child[consistency] {
						node[concept] {Consistency} [counterclockwise from=0] {
								% child {node[concept] {1}}
								% child {node[concept] {2}}
								% child {node[concept] {3}}
								% child {node[concept] {4}}
							}
					}
				child[interactivity] {
						node[concept] {Interactivity} [counterclockwise from=-90] {
								% child {node[concept] {1}}
								% child {node[concept] {2}}
								% child {node[concept] {3}}
								% child {node[concept] {4}}
							}
					}
			};
		\end{tikzpicture}
	}
\end{Frame}

\begin{Frame}{Mindmap (Taxonomy)}
	\tikzset{
		key/.style={
				draw=OrangeRed,
			},
		convention/.style={
				draw=Cyan,
			},
		trick/.style={
				draw=YellowGreen,
			},
	}
	\resizebox{0.85\textwidth}{!}{
		\begin{forest}
			for tree={my tree}
			[
			MonoGS
				[
					Visual Odometry,for children={visible on=<1->}
						[
							Tracking,for children={visible on=<2->}
								[
									Inverse Rendering,font=\bf,for children={visible on=<3->}
										[
											Analytical Jacobians Derivation,key
										]
										[
											Photometric Appearance \& Depth Loss,convention
										]
										[
											Optimizable Exposure,trick
										]
										[
											Pixel-wise Penalization,trick
										]
								]
						]
						[
							Mapping,for children={visible on=<2->}
								[
									Bundle Adjustment,font=\bf,for children={visible on=<3->}
										[
											Photometric Appearance \& Depth Loss,convention
										]
										[
											Isotropic Regularization,key
										]
										[
											Random Recall,trick
										]
								]
						]
				]
				[
					Online Pipeline,for children={visible on=<1->}
						[
							Keyframe Management,for children={visible on=<2->}
								[
									Registration,for children={visible on=<3->}
										[
											Gaussian Covisibility,key
										]
										[
											Relative Translation,convention
										]
								]
								[
									Removal,for children={visible on=<3->}
										[
											Gaussian Overlap Coefficient,key
										]
								]
						]
						[
							Gaussian Management,for children={visible on=<2->}
								[
									Insertion,for children={visible on=<3->}
										[
											Keyframing,convention
										]
								]
								[
									Pruning,for children={visible on=<3->}
										[
											Gaussian Covisibility,key
										]
								]
						]
				]
			]
		\end{forest}
	}
	\resizebox{0.08\textwidth}{!}{
		\begin{tikzpicture}[visible on=<3->]
			\node [my node for tree, draw=YellowGreen] (trick node) {trick};
			\node [my node for tree, draw=OrangeRed, above of = trick node] {key method};
			\node [my node for tree, draw=Cyan, below of = trick node] (convention) {convention};
		\end{tikzpicture}
	}
	\blfootnote{powered by \texttt{forest} package}
	\blfootnote{thanks to \href{https://tex.stackexchange.com/a/556837/240783}{Drawing Taxonomy Diagram in Latex}}
\end{Frame}

\subsection{Timeline}
\begin{Frame}{Timeline}
	\centering
	\begin{tikzpicture}
		\pgfmathsetmacro{\dy}{0.3cm/1pt};
		\pgfmathsetmacro{\nodeNum}{3};
		\pgfmathsetmacro{\sepNum}{\nodeNum-1};
		\pgfmathsetmacro{\length}{0.8*\linewidth/1pt};
		\pgfmathsetmacro{\edgeLength}{1cm/1pt};
		\pgfmathsetmacro{\dx}{(\length-\edgeLength-\edgeLength)/\sepNum};
		\pgfmathsetmacro{\xShift}{0};
		\tikzset{
			note/.style={
					anchor=north, align=center, text width=\dx, yshift=-\dy/3, font={\scriptsize}
				},
			time/.style={
					anchor=south, font={\scriptsize\bf}
				},
		};

		\coordinate (start) at (0 pt,0 pt);
		\coordinate (end) at (\length pt,0 pt);
		\draw [line width=1.5pt,-stealth] (start) -- (end);

		\foreach \counter in {0,...,\sepNum} {
				\coordinate (s\counter) at (\edgeLength+\counter*\dx pt,0);
				\coordinate (t\counter) at ($(s\counter)+(0,\dy pt)$);
				\draw [line width=1.5pt] (s\counter) -- (t\counter);
			}

		\node [time] at (t0.north) {
			11/2023 - 12/2024
		};
		\node [note] at (s0.south) {
			LEGS~\autocite{shiLanguageEmbedded3D2023nov}\\[1ex]
			LangSplat~\autocite{qinLangSplat3DLanguage2023}\\[1ex]
			Feature 3DGS~\autocite{zhouFeature3DGSSupercharging2024apr}\\[1ex]
			Gaussian Grouping~\autocite{yeGaussianGroupingSegment2023}\\[1ex]
			Segment Any Gaussian~\autocite{cenSegmentAny3D2023}\\[1ex]
		};
		\node [time] at (t1.north) {
			01/2024 - 04/2024
		};
		\node [note] at (s1.south) {
			CoSSegGaussians~\autocite{douCoSSegGaussiansCompactSwift2024jan}\\[1ex]
			Semantic Gaussians~\autocite{guoSemanticGaussiansOpenVocabulary2024mar}\\[1ex]
			Feature Splating~\autocite{qiuFeatureSplattingLanguageDriven2024apr}\\[1ex]
			CLIP-GS~\autocite{liaoCLIPGSCLIPInformedGaussian2024apr}
		};
		\node [time] at (t2.north) {
			05/2024 - 06/2024
		};
		\node [note] at (s2.south) {
			GOI~\autocite{quGOIFind3D2024may}\\[1ex]
			RT-GS2~\autocite{jurcaRTGS2RealTimeGeneralizable2024may}\\[1ex]
			SA-GS~\autocite{xiongSAGSSemanticAwareGaussian2024may}\\[1ex]
			Fast-LGS~\autocite{jiFastLGSSpeedingLanguage2024jun}
		};
	\end{tikzpicture}
\end{Frame}

\appendix

\section{\appendixname}

\subsection{References}

\begin{frame}[allowframebreaks]
	\frametitle{References}
	\nocite{*}
	\printbibliography[heading=none]
\end{frame}

\end{document}
