%%%%%%%%%%%%%%%%%%%%%%%%%%%%%%%%%%%%%%%%%%%%%%%%%%%%%%%%%%%%%%%%%%%%%%%%%%%%%%%%
%%%%%%%%%%%%%%%%%%%%%%%%%%%%%%%%%%%%% TODO %%%%%%%%%%%%%%%%%%%%%%%%%%%%%%%%%%%%%
%%%%%%%%%%%%%%%%%%%%%%%%%%%%%%%%%%%%%%%%%%%%%%%%%%%%%%%%%%%%%%%%%%%%%%%%%%%%%%%%
% - Add: AGWN 
% - Add: AGWN 
% - Update: Put corollaries into "Derivation of the optimal Kalman gain"
% - Add: corollaries into "Derivation of the optimal Kalman gain"
% - Update: put "some tricks" and "Joseph form" into corollaries
% - Update: All the spacings based on "em"

%%%%%%%%%%%%%%%%%%%%%%%%%%%%%%%%%%%%%%%%%%%%%%%%%%%%%%%%%%%%%%%%%%%%%%%%%%%%%%%%
%%%%%%%%%%%%%%%%%%%%%%%%%%%%%%%%%%% PREAMBLE %%%%%%%%%%%%%%%%%%%%%%%%%%%%%%%%%%%
%%%%%%%%%%%%%%%%%%%%%%%%%%%%%%%%%%%%%%%%%%%%%%%%%%%%%%%%%%%%%%%%%%%%%%%%%%%%%%%%
\documentclass[utf-8, 10pt, aspectratio=1610]{beamer}
\mode<presentation>
\usetheme[
background=light, 
titleformat=regular,
subsectionpage=progressbar,
block=fill,
]{metropolis}
\usepackage{appendixnumberbeamer}

\usepackage{xcolor}
    % ref: http://zhongguose.com
    \definecolor{kongquelan}{RGB}{14,176,201}
    \definecolor{koushaolv}{RGB}{93,190,138}
    \definecolor{yingwulv}{RGB}{91,174,35}
    \definecolor{shenhuilan}{RGB}{19,44,51}
    \definecolor{jianniaolan}{RGB}{20,145,168}
    \definecolor{jiguanghong}{RGB}{243,59,31}
    \definecolor{xiangyehong}{RGB}{240,124,130}

% typefaces
\usepackage{fontawesome5}

\setmainfont{Source Serif 4}
\usepackage{xeCJK}
    \setCJKmainfont{思源宋体}
    \setCJKsansfont{思源黑体}
    \setCJKmonofont{思源等宽}

% tikz_annotated_equations
\usepackage{tikz}
\usetikzlibrary{calc,tikzmark}
\usepackage{tcolorbox}
    \colorlet{blue_marknode_color}{kongquelan!50}
    \colorlet{blue_annotate_color}{kongquelan}
    \colorlet{green_marknode_color}{yingwulv!50}
    \colorlet{green_annotate_color}{yingwulv}
    \colorlet{red_marknode_color}{jiguanghong!40}
    \colorlet{red_annotate_color}{jiguanghong!80}

% maths
\usepackage{amsmath}
\usepackage{amsthm}
\usepackage{amsfonts}
\usepackage{mathtools}
\usepackage{cases}
\usepackage{empheq}

% bibliography
\usepackage[backend=biber,natbib=true,style=ieee]{biblatex}
\addbibresource{references.bib}
\renewcommand*{\bibfont}{\normalfont\small}

% table
\usepackage{booktabs}
\usepackage{tabularx}

% table
\usepackage{hyperref}
\hypersetup{
    colorlinks=true,
    linkcolor=.,
    anchorcolor=.,
    filecolor=.,
    menucolor=.,
    runcolor=.,
    citecolor=.,
    urlcolor=jianniaolan,
}

%%%%%%%%%%%%%%%%%%%%%%%%%%%%%%%%%%%%%%%%%%%%%%%%%%%%%%%%%%%%%%%%%%%%%%%%%%%%%%%%
%%%%%%%%%%%%%%%%%%%%%%%%%%%%%%%%%%%%% INFO  %%%%%%%%%%%%%%%%%%%%%%%%%%%%%%%%%%%%
%%%%%%%%%%%%%%%%%%%%%%%%%%%%%%%%%%%%%%%%%%%%%%%%%%%%%%%%%%%%%%%%%%%%%%%%%%%%%%%%
\title{Kalman Filter}
\subtitle{A rigorous but painless introduction for Roboticists}
\author{shuqi}
\date{\today}
\institute{
    \faGithub\;
    \href{https://github.com/xiaosq2000}{xiaosq2000}
    \quad
    \faEnvelope\;
    \href{xiaosq2000@gmail.com}{xiaosq2000@gmail.com}
}

%%%%%%%%%%%%%%%%%%%%%%%%%%%%%%%%%%%%%%%%%%%%%%%%%%%%%%%%%%%%%%%%%%%%%%%%%%%%%%%%
%%%%%%%%%%%%%%%%%%%%%%%%%%%%%%%%%%% DOCUMENT %%%%%%%%%%%%%%%%%%%%%%%%%%%%%%%%%%%
%%%%%%%%%%%%%%%%%%%%%%%%%%%%%%%%%%%%%%%%%%%%%%%%%%%%%%%%%%%%%%%%%%%%%%%%%%%%%%%%
\begin{document}

\begin{frame}
	\titlepage
\end{frame}

\begin{frame}{TL; DR}
	\par Kalman filter is an optimal, recursive, linear-quardratic estimator\footnote{alias \alert{LQE}} for dynamic systems under certain assumptions.
	\vspace*{\fill}

	\begin{quote}
		In summary, the following assumptions are made about random processes: Physical random phenomena may be thought of as due to primary random sources exciting dynamic systems. The primary sources are assumed to be independent Gaussian random processes with zero mean; the dynamic systems will be linear.\supercite{kalman_new_1960}
		\flushright ---\;\textrm{Rudolf\ E.\ Kálmán}
	\end{quote}
\end{frame}

\begin{frame}{Prerequistes}
	Undergrate-level
	\begin{itemize}
		\item Probability and Stochasitic Process
		\item Linear Algebra
		\item Matrix Caculus
	\end{itemize}
\end{frame}

\begin{frame}{Outline}
	\setbeamertemplate{section in toc}[sections numbered]
	\tableofcontents[hideallsubsections]
\end{frame}

\section{Underlying dynamic system model}

\begin{frame}[allowframebreaks]{Underlying dynamic system model}
	\vspace*{\fill}
	A classic state-space representation of dynamic system\supercite{enwiki:1136534469}.
	\begin{figure}[htbp]
		\vspace*{2.5em}
		\begin{numcases}{}
			\dot{\mathbf{x}}(\tikzmarknode{time}{\colorbox{blue_marknode_color}{\(t\)}}) = \tikzmarknode{system_matrix}{\colorbox{green_marknode_color}{\(\mathbf{A}\)}}(t) \tikzmarknode{state_vector}{\colorbox{blue_marknode_color}{\(\mathbf{x}\)}}(t) + \tikzmarknode{control_matrix}{\colorbox{green_marknode_color}{\(\mathbf{B}\)}}(t) \tikzmarknode{input_vector}{\colorbox{blue_marknode_color}{\(\mathbf{u}\)}}(t) \\
			\tikzmarknode{output_vector}{\colorbox{blue_marknode_color}{\(\mathbf{y}\)}}(t) = \tikzmarknode{measurement_matrix}{\colorbox{green_marknode_color}{\(\mathbf{C}\)}}(t) \mathbf{x}(t) + \tikzmarknode{feedforward_matrix}{\colorbox{green_marknode_color}{\(\mathbf{D}\)}}(t) \mathbf{u}(t)
		\end{numcases}
		\begin{tikzpicture}[overlay,remember picture,>=stealth,nodes={align=left,inner ysep=1pt},<-]
			\path (time.north) ++ (0em,0.5em) node[anchor=south east,color=blue_annotate_color] (time_annotate) {\footnotesize{\(\in \mathbb{R}\), time}};
			\draw [color=blue_annotate_color] (time.north) |- (time_annotate.south west);

			\path (system_matrix.north) ++ (0,2em) node[anchor=south east,color=green_annotate_color] (system_green_annotate) {\footnotesize{\(\in \mathcal{M}(n,n)\), system matrix }};
			\draw [color=green_annotate_color] (system_matrix.north) |- (system_green_annotate.south west);

			\path (state_vector.north) ++ (0em,3.5em) node[anchor=south east,color=blue_annotate_color] (state_blue_annotate) {\footnotesize{\(\in \mathbb{R}^n\), state vector}};
			\draw [color=blue_annotate_color] (state_vector.north) |- (state_blue_annotate.south west);

			\path (control_matrix.north) ++ (0em,2.8em) node[anchor=south west,color=green_annotate_color] (control_green_annotate) {\footnotesize{\(\in \mathcal{M}(n,p)\), input/control matrix}};
			\draw [color=green_annotate_color] (control_matrix.north) |- (control_green_annotate.south east);

			\path (input_vector.north) ++ (0em,1.5em) node[anchor=south west,color=blue_annotate_color] (input_blue_annotate) {\footnotesize{\(\in \mathbb{R}^p\), input/control vector}};
			\draw [color=blue_annotate_color] (input_vector.north) |- (input_blue_annotate.south east);

			\path (output_vector.south) ++ (0em,-1.5em) node[anchor=north east,color=blue_annotate_color] (output_blue_annotate) {\footnotesize{\(\in \mathbb{R}^q\), output/measurement vector}};
			\draw [color=blue_annotate_color] (output_vector.south) |- (output_blue_annotate.south west);

			\path (measurement_matrix.south) ++ (0em,-2.8em) node[anchor=north west,color=green_annotate_color] (measurement_green_annotate) {\footnotesize{\(\in \mathcal{M}(q,n)\), output/measurement matrix}};
			\draw [color=green_annotate_color] (measurement_matrix.south) |- (measurement_green_annotate.south east);

			\path (feedforward_matrix.south) ++ (0em,-1.5em) node[anchor=north west,color=green_annotate_color] (feedforward_green_annotate) {\footnotesize{\(\in \mathcal{M}(q,p)\), feedforward matrix}};
			\draw [color=green_annotate_color] (feedforward_matrix.south) |- (feedforward_green_annotate.south east);
		\end{tikzpicture}
		\vspace*{2.5em}
		\caption{continuous-time, deterministic, time-variant, linear dynamic system}
	\end{figure}
	\vspace*{\fill}

	\framebreak
	\vspace*{\fill}
	\begin{definition}[the underlying model of Kalman filter]
		A discrete-time, linear dynamic system\footnote{the feedforward part is omitted for convenience}, with additive Gaussian white noise\supercite{enwiki:1164245413}.
		\vspace*{1em}
		\begin{figure}
			\begin{numcases}{}
				\mathbf{x}_{k} =  \mathbf{A}\tikzmarknode{index}{\colorbox{blue_marknode_color}{\(_k\)}} \mathbf{x}_{k-1} + \mathbf{B}_k\mathbf{u}_{k} + \tikzmarknode{process_noise}{\colorbox{red_marknode_color}{\(\mathbf{v}_k\)}} \\ \label{eq:state-transition-eqation}
				\mathbf{y}_{k} = \mathbf{C}_k\mathbf{x}_k + \tikzmarknode{measurement_noise}{\colorbox{red_marknode_color}{\(\mathbf{w}_k\)}} \label{eq:observation-eqation}
			\end{numcases}
			\begin{tikzpicture}[overlay,remember picture,>=stealth,nodes={align=left,inner ysep=1pt},<-]
				\path (index.north) ++ (0em,1em) node[anchor=south east,color=blue_annotate_color] (index_annotate) {\footnotesize{\(\in \mathbb{N}\), index/timestamp}};
				\draw [color=blue_annotate_color] (index.north) |- (index_annotate.south west);
				\path (process_noise.north) ++ (0em,1em) node[anchor=south west,color=red_annotate_color] (process_red_annotate) {\footnotesize{\(\sim \mathcal{N}(\mathbf{0}, \mathbf{Q}_k)\), process noise}};
				\draw [color=red_annotate_color] (process_noise.north) |- (process_red_annotate.south east);
				\path (measurement_noise.south) ++ (0em,-0.5em) node[anchor=north west,color=red_annotate_color] (measurement_red_annotate) {\footnotesize{\(\sim \mathcal{N}(\mathbf{0}, \mathbf{R}_k)\), measurement noise}};
				\draw [color=red_annotate_color] (measurement_noise.south) |- (measurement_red_annotate.south east);
			\end{tikzpicture}
			\vspace*{1em}
			\caption{discrete-time, random, time-variant, linear dynamic system}
		\end{figure}
	\end{definition}
	\vspace*{\fill}

	\framebreak
	\begin{corollary}[]
		\begin{equation}
			\forall\,k \in \mathbb{N} \left(\mathbf{x}_k \sim \mathcal{N}, \mathbf{y}_k\sim \mathcal{N} \right)
		\end{equation}
		\begin{minipage}[t]{0.48\textwidth}
			\vspace*{-2em}
			\begin{numcases}{}
				\operatorname{E}\left(\mathbf{x}_k\right) = \mathbf{A}_k \operatorname{E}\left( \mathbf{x}_{k-1} \right) + \mathbf{B}_k \mathbf{u}_k \\
				\operatorname{Cov}\left(\mathbf{x}_k\right) = \mathbf{A}_k \operatorname{Cov}\left(\mathbf{x}_{k-1}\right) \mathbf{A}_k ^{\mathrm{T}} + \mathbf{Q}_k
			\end{numcases}
		\end{minipage}
		\hfill
		\begin{minipage}[t]{0.48\textwidth}
			\vspace*{-2em}
			\begin{numcases}{}
				\operatorname{E}\left(\mathbf{y}_k\right) = \mathbf{C}_k \operatorname{E}\left( \mathbf{x}_{k} \right) \\
				\operatorname{Cov}\left(\mathbf{y}_k\right) = \mathbf{C}_k \operatorname{Cov}\left(\mathbf{x}_{k}\right) \mathbf{C}_k ^{\mathrm{T}} + \mathbf{R}_k
			\end{numcases}
		\end{minipage}
	\end{corollary}
\end{frame}

\section{Linear quardratic estimation}

\begin{frame}{Problem formulation}
	Given the state-space dynamic model and
	\begin{table}
		\begin{tabular}{cc}
			\toprule
			\(\hat{\mathbf{x}}_0\)                     & prior estimated distribution of the initial state \\
			\(\mathbf{Q}_0, \cdots, \mathbf{Q}_{k-1}\) & prior knowledge of the process noise              \\
			\(\mathbf{R}_0, \cdots, \mathbf{R}_{k-1}\) & prior knowledge of the measurement noise          \\
			\(\mathbf{u}_0, \cdots, \mathbf{u}_{k-1}\) & control signals (deterministic) up to now         \\
			\(\mathbf{y}_0, \cdots, \mathbf{y}_{k-1}\) & measurements up to now                            \\
			\bottomrule
		\end{tabular}
	\end{table}
	to estimate \(\mathbf{x}_{k}\) and its uncertainties in a linear and recursive method.
\end{frame}

\begin{frame}[allowframebreaks]{Predict \& Correct}
	The Kalman filter is usually conceptualized as two distinct phases, of which names are enough enlightening.
	\begin{definition}[state estimator in the Kalman filter]
		\begin{figure}[htbp]
			\begin{numcases}{}
				\tikzmarknode{predicted_state}{\colorbox{red_marknode_color}{\(\hat{\mathbf{x}}_{k \vert k-1}\)}} = \mathbf{A}_k \hat{\mathbf{x}}_{k-1\vert k-1} + \mathbf{B}_{k} \mathbf{u}_k& \small predict\\ \label{eq:state-estimate-update-equation}
				\hat{\mathbf{x}}_{k \vert k} = \hat{\mathbf{x}}_{k \vert k-1} + \tikzmarknode{gain}{\colorbox{green_marknode_color}{\(\mathbf{K}_k\)}} \left(\tikzmarknode{innovation}{\colorbox{blue_marknode_color}{\(\mathbf{y}_{k} - \mathbf{C}_{k} \hat{\mathbf{x}}_{k|k-1}\)}}\right) & \small correct/update
			\end{numcases}
			\begin{tikzpicture}[overlay,remember picture,>=stealth,nodes={align=left,inner ysep=1pt},<-]
				\path (predicted_state.north) ++ (0em,1em) node[anchor=south west,color=red_annotate_color] (predicted_state_annotate) {\footnotesize{estimate of \(\mathbf{x}\) at \(k\), given \(k-1\) measurements}};
				\draw [color=red_annotate_color] (predicted_state.north) |- (predicted_state_annotate.south east);
				\path (gain.south) ++ (0em,-1em) node[anchor=north east,color=green_annotate_color] (gain_annotate) {\footnotesize{Kalman gain}};
				\draw [color=green_annotate_color] (gain.south) |- (gain_annotate.south west);
				\path (innovation.south) ++ (0em,-1em) node[anchor=north west,color=blue_annotate_color] (innovation_annotate) {\footnotesize{\(=: \tilde{y}_{k\vert k-1} \), innovation/measurement pre-fit residual}};
				\draw [color=blue_annotate_color] (innovation.south) |- (innovation_annotate.south east);
			\end{tikzpicture}
			\vspace*{1em}
			\caption{equations to update state estimates}
		\end{figure}
	\end{definition}

	\framebreak
	\vspace*{\fill}
	\begin{block}{Remark (motivation of prediction)}
		It is straightforward and reasonable that the predict-phase state update (eq.~\ref{eq:state-estimate-update-equation}) is derived from the state transition (eq.~\ref{eq:state-transition-eqation}), by replacing the expectation, \(\operatorname{E}\left(\square\right)\) with the estimate, \(\hat{\square}\).
	\end{block}
	\begin{block}{Remark (motivation of correction)}
		Intuitively, the \textcolor{green_annotate_color}{Kalman gain} is like a weight coefficient to emphasize or weaken the information from real measurements, and the weighted difference between prediction and measurements (measurement pre-fit residual) is added to correct the prediction.
		\vspace{1em}
		\par The \textcolor{green_annotate_color}{Kalman gain} certainly cannot be chosen arbitrarily. How? Stay tuned.
	\end{block}
	\vspace*{\fill}

	\framebreak
	\vspace*{\fill}
	\begin{corollary}[ideal invariants]
		Our prior knowledge is assumed to be \(100\%\) accurate, not only the model, but also the initial estimate (\( \hat{\mathbf{x}}_0 = \mathbf{x}_0 \)), then there are some invariants during the whole process, e.g., \(\forall\,k\in \mathbb{N},\ \)
		\begin{align}
			\operatorname{E}\left(\tilde{\mathbf{x}}_{k\vert k-1}\right) & = \mathbf{0}                                         \\
			\operatorname{E}\left(\tilde{\mathbf{y}}_{k\vert k-1}\right) & = \mathbf{0}                                         \\
			\operatorname{E}\left(\tilde{\mathbf{x}}_{k\vert k}\right)   & = \mathbf{0} \label{eq:expectation-of-error-is-zero}
		\end{align}
	\end{corollary}
	\vspace*{\fill}

	\framebreak
	\vspace*{\fill}
	\begin{corollary}[prior error covariance update]
		\begin{figure}[htbp]
			\vspace*{0.3em}
			\begin{equation}
				\tikzmarknode{prior_error_covariance}{\colorbox{blue_marknode_color}{\(\tilde{\mathbf{P}}_{k\vert k-1}\)}}                                               = \mathbf{A}_k \tikzmarknode{last_posterior_error_covariance}{\colorbox{green_marknode_color}{\(\tilde{\mathbf{P}}_{k-1\vert k-1}\)}} \mathbf{A}_k ^{\mathrm{T}} + \mathbf{R}_k
			\end{equation}
			\begin{tikzpicture}[overlay,remember picture,>=stealth,nodes={align=left,inner ysep=1pt},<-]
				\path (prior_error_covariance.south) ++ (0em,-0.5em) node[anchor=north west,color=blue_annotate_color] (prior_error_covariance_annotate) {\footnotesize{\(:=\operatorname{Cov}\left(\hat{\mathbf{x}}_{k\vert k-1} - \mathbf{x}_{k}\right)\), prior error covariance}};
				\draw [color=blue_annotate_color] (prior_error_covariance.south) |- (prior_error_covariance_annotate.south east);
				\path (last_posterior_error_covariance.north) ++ (0em,1em) node[anchor=south east,color=green_annotate_color] (last_posterior_error_covariance_annotate) {\footnotesize{\(\operatorname{Cov}\left(\hat{\mathbf{x}}_{k-1\vert k-1} - \mathbf{x}_{k-1}\right)\), last posterior error covariance}};
				\draw [color=green_annotate_color] (last_posterior_error_covariance.north) |- (last_posterior_error_covariance_annotate.south west);
			\end{tikzpicture}
		\end{figure}
	\end{corollary}
	\begin{proof}
		\begin{align}
			\hat{\mathbf{x}}_{k\vert k-1} - \mathbf{x}_{k}                                & = \mathbf{A}_{k} \hat{\mathbf{x}}_{k-1 \vert k-1} + \mathbf{B}_k \mathbf{u}_k  - \left(\mathbf{A}_k \mathbf{x}_{k-1} + \mathbf{B}_k \mathbf{u}_k +\mathbf{v}_k\right)     \\
			\operatorname{Cov}\left(\hat{\mathbf{x}}_{k\vert k-1} - \mathbf{x}_{k}\right) & = \mathbf{A}_k \operatorname{Cov}\left(\hat{\mathbf{x}}_{k-1\vert k-1} - \mathbf{x}_{k-1}\right) \mathbf{A}_k ^{\mathrm{T}} + \operatorname{Cov}\left(\mathbf{v}_k\right) \\
			\tilde{\mathbf{P}}_{k\vert k-1}                                               & = \mathbf{A}_k \tilde{\mathbf{P}}_{k-1\vert k-1} \mathbf{A}_k ^{\mathrm{T}} + \mathbf{R}_k
		\end{align}
	\end{proof}
	\vspace*{\fill}

	% \framebreak
	% \vspace*{\fill}
	% \begin{definition}[state uncertainty estimator in the predict-phase]
	% 	\begin{figure}[htbp]
	% 		\vspace*{-1em}
	% 		\begin{numcases}{}
	% 			\hat{\operatorname{E}}\left(\mathbf{x}_{k\vert k-1}\right) = \mathbf{A}_k \hat{\operatorname{E}}\left(\mathbf{x}_{k-1\vert k-1}\right)  + \mathbf{B}_{k} \mathbf{u}_k\\
	% 			\tikzmarknode{prior_covariance}{\colorbox{blue_marknode_color}{\(\hat{\operatorname{Cov}}\left(\mathbf{x}_{k\vert k-1}\right)\)}} = \mathbf{A}_k \tikzmarknode{initial_covariance}{\colorbox{blue_marknode_color}{\(\hat{\operatorname{Cov}}\left(\mathbf{x}_{k-1\vert k-1}\right)\)}}\mathbf{A}_k ^{\mathrm{T}} + \mathbf{Q}_k
	% 		\end{numcases}
	% 		\begin{tikzpicture}[overlay,remember picture,>=stealth,nodes={align=left,inner ysep=1pt},<-]
	% 			\path (prior_covariance.south) ++ (0em,-2em) node[anchor=north west,color=blue_annotate_color] (prior_covariance_annotate) {\footnotesize{\(=: \hat{\mathbf{P}}_{k\vert k-1}\), prior/predicted covariance}};
	% 			\draw [color=blue_annotate_color] (prior_covariance.south) |- (prior_covariance_annotate.south east);
	% 			\path (initial_covariance.south) ++ (0em,-0.5em) node[anchor=north west,color=blue_annotate_color] (initial_covariance_annotate) {\footnotesize{\(=: \hat{\mathbf{P}}_{k-1\vert k-1}\), initial/prior\footnote{abuse of ``prior'', so pay attention to context} covariance}};
	% 			\draw [color=blue_annotate_color] (initial_covariance.south) |- (initial_covariance_annotate.south east);
	% 		\end{tikzpicture}
	% 		\vspace*{1.5em}
	% 		\caption{the prior covariance update equation in the predict-phase}
	% 	\end{figure}
	% \end{definition}
	% \vspace*{\fill}

	% \framebreak
	% \vspace*{\fill}
	% \begin{block}{Remark}
	% 	Be cautious. It's easy to confuse ``the estimator of the covariance'' and ``the covariance of the estimator''. Take the predict-phase for example, and pay attention to the place of the hat mark.
	% 	\begin{figure}[htbp]
	% 		\vspace*{-2em}
	% 		\begin{numcases}{}
	% 			\operatorname{E}\left(\hat{\mathbf{x}}_{k\vert k-1}\right) = \mathbf{A}_k \operatorname{E}\left(\hat{\mathbf{x}}_{k-1\vert k-1}\right)  + \mathbf{B}_{k} \mathbf{u}_k \\
	% 			\operatorname{Cov}\left(\hat{\mathbf{x}}_{k\vert k-1}\right) = \mathbf{A}_k \operatorname{Cov}\left(\hat{\mathbf{x}}_{k-1\vert k-1}\right)\mathbf{A}_k ^{\mathrm{T}}
	% 		\end{numcases}
	% 		\caption{the expectation/covariance of the estimator}
	% 	\end{figure}
	% \end{block}
	% \vspace*{\fill}

	\framebreak
	\vspace*{\fill}
	\begin{corollary}[innovation covariance]
		\begin{figure}[htbp]
			\begin{equation}
				\tikzmarknode{innovation_covariance}{\colorbox{red_marknode_color}{\(\mathbf{S}_{k\vert k-1}\)}} = \mathbf{C}_k \tikzmarknode{prior_error_covariance}{\colorbox{blue_marknode_color}{\(\tilde{\mathbf{P}}_{k\vert k-1}\)}} \mathbf{C}_k ^{\mathrm{T}} + \mathbf{R}_k
			\end{equation}
			\begin{tikzpicture}[overlay,remember picture,>=stealth,nodes={align=left,inner ysep=1pt},<-]
				\path (innovation_covariance.south) ++ (0em,-1em) node[anchor=north west,color=red_annotate_color] (innovation_covariance_annotate) {\footnotesize{\(:=\operatorname{Cov}\left(\tilde{\mathbf{y}}_{k\vert k-1}\right)\)}};
				\draw [color=red_annotate_color] (innovation_covariance.south) |- (innovation_covariance_annotate.south east);
			\end{tikzpicture}
		\end{figure}
	\end{corollary}
	\begin{proof}
		\begin{align}
			\operatorname{Cov}\left(\tilde{\mathbf{y}}_{k\vert k-1}\right) & = \operatorname{Cov}\left(\mathbf{C}_k \mathbf{x}_k + \mathbf{w}_k - \mathbf{C}_k \hat{\mathbf{x}}_{k\vert k-1}\right)                                                \\
			                                                               & = \mathbf{C}_k \operatorname{Cov}\left( \mathbf{x}_k - \hat{\mathbf{x}}_{k\vert k-1} \right) \mathbf{C}_k ^{\mathrm{T}} + \operatorname{Cov}\left(\mathbf{w}_k\right) \\
			                                                               & = \mathbf{C}_k \tilde{\mathbf{P}}_{k\vert k-1} \mathbf{C}_k ^{\mathrm{T}} + \mathbf{R}_k
		\end{align}
	\end{proof}
	\vspace*{\fill}
\end{frame}

\section{Derivation of the optimal Kalman gain}
\begin{frame}[allowframebreaks]{Derivation of the optimal Kalman gain}
	\vspace*{\fill}
	\par As mentioned above, we've already designed an estimator with a well-formed structure, except that the parameter \(\mathbf{K}_k\) hasn't been decided yet.
	\vspace*{\fill}
	\begin{definition}[the optimal Kalman gain]
		\par \alert{MMSE} (minimum mean square error) is chosen as our rule for optimality, i.e.,
		\begin{figure}[htbp]
			\vspace*{-2em}
			\begin{equation}
				\mathbf{K}_k := \underset{\mathbf{K}_k}{\operatorname{argmin}} \operatorname{E} \left(\Vert \tikzmarknode{error}{\colorbox{red_marknode_color}{\(\tilde{\mathbf{x}}_k\)}} \Vert_2\right)
			\end{equation}
			\begin{tikzpicture}[overlay,remember picture,>=stealth,nodes={align=left,inner ysep=1pt},<-]
				\path (error.south) ++ (0em,-1em) node[anchor=north east,color=red_annotate_color] (error_annotate) {\footnotesize{\(=: \hat{\mathbf{x}}_k - \mathbf{x}_k\), error}};
				\draw [color=red_annotate_color] (error.south) |- (error_annotate.south west);
			\end{tikzpicture}
			\caption{minimum expectation of L2-norm of the error random variable from a Bayesian approach}
		\end{figure}
	\end{definition}
	\vspace*{\fill}

	\framebreak
	\vspace*{\fill}
	Some little tricks here,
	\begin{equation}
		\underset{\mathbf{K}_k}{\operatorname{argmin}} \operatorname{E} \left(\Vert \tilde{\mathbf{x}}_k \Vert_2\right) = \underset{\mathbf{K}_k}{\operatorname{argmin}} \operatorname{E} \left(\Vert \tilde{\mathbf{x}}_k \Vert_2^2 \right)
	\end{equation}
	\begin{align}
		\operatorname{E} \left(\Vert \tilde{\mathbf{x}}_k \Vert_2^2\right) & = \operatorname{E} \left( \tilde{\mathbf{x}}_k ^{\mathrm{T}} \tilde{\mathbf{x}}_k \right)                                \\
		                                                                   & = \operatorname{E} \left( \operatorname{Tr} \left(\tilde{\mathbf{x}}_k^{\mathrm{T}} \tilde{\mathbf{x}}_k \right) \right) \\
		                                                                   & = \operatorname{E} \left( \operatorname{Tr} \left(\tilde{\mathbf{x}}_k \tilde{\mathbf{x}}_k^{\mathrm{T}} \right) \right) \\
		                                                                   & = \operatorname{Tr} \left( \operatorname{E} \left(\tilde{\mathbf{x}}_k \tilde{\mathbf{x}}_k^{\mathrm{T}} \right) \right)
	\end{align}
	Since eq.~\ref{eq:expectation-of-error-is-zero},
	\begin{figure}[htbp]
		\vspace*{-3em}
		\begin{equation}
			\operatorname{E} \left(\tilde{\mathbf{x}}_k \tilde{\mathbf{x}}_k^{\mathrm{T}} \right) = \tikzmarknode{error_covariance}{\colorbox{green_marknode_color}{\(\operatorname{Cov}\left(\tilde{\mathbf{x}}_k\right)\)}}
		\end{equation}
		\begin{tikzpicture}[overlay,remember picture,>=stealth,nodes={align=left,inner ysep=1pt},<-]
			\path (error_covariance.north) ++ (0em,1em) node[anchor=south west,color=green_annotate_color] (error_covariance_annotate) {\footnotesize{\(:= \tilde{\mathbf{P}}_{k}\ \text{or}\ \tilde{\mathbf{P}}_{k\vert k}\), posterior error covariance }};
			\draw [color=green_annotate_color] (error_covariance.north) |- (error_covariance_annotate.south east);
		\end{tikzpicture}
	\end{figure}
	\vspace*{\fill}

	\framebreak
	The optimal Kalman gain becomes the solution of
	\begin{equation}
		\frac{\partial}{\partial \mathbf{K}_k} \operatorname{Tr} \left(\tilde{\mathbf{P}}_k\right) = \mathbf{0}
	\end{equation}

	Hold on! I know you can't wait to roll up your sleeves, expand out the terms and play with the messy matrix calculus, but the following \alert{Joseph form} of the posterior error covariance will save you a lot of trouble.

	\framebreak
	\begin{align}
		\tilde{\mathbf{P}}_k & = \operatorname{Cov}\left(\hat{\mathbf{x}}_k - \mathbf{x}_k \right)                                                                                                                                                                                                                                                                                           \\
		                     & = \operatorname{Cov}\left(\hat{\mathbf{x}}_{k\vert k-1} + \mathbf{K}_k \left(\mathbf{y}_k - \mathbf{C}_k \hat{\mathbf{x}}_{k\vert k-1}\right) - \mathbf{x}_k \right)                                                                                                                                                                                          \\
		                     & = \operatorname{Cov}\left(\hat{\mathbf{x}}_{k\vert k-1} + \mathbf{K}_k \left(\mathbf{C}_k \mathbf{x}_k + \mathbf{w}_k - \mathbf{C}_k \hat{\mathbf{x}}_{k\vert k-1}\right) - \mathbf{x}_k \right)                                                                                                                                                              \\
		                     & = \operatorname{Cov}\left( \left(\mathbf{I} - \mathbf{K}_k \mathbf{C}_k\right) \hat{\mathbf{x}}_{k \vert k-1} - \left( \mathbf{I} - \mathbf{K}_k \mathbf{C}_k \right) \mathbf{x}_k + \mathbf{K}_k \mathbf{w}_k \right)                                                                                                                                        \\
		                     & = \operatorname{Cov}\left( \left(\mathbf{I} - \mathbf{K}_k \mathbf{C}_k\right) \hat{\mathbf{x}}_{k \vert k-1} - \left( \mathbf{I} - \mathbf{K}_k \mathbf{C}_k \right) \mathbf{x}_k \right) + \operatorname{Cov} \left( \mathbf{K}_k \mathbf{w}_k \right)                                                                                                      \\
		                     & = \tikzmarknode{joseph_form}{\colorbox{blue_marknode_color}{\(\left(\mathbf{I} - \mathbf{K}_k \mathbf{C}_k\right) \tilde{\mathbf{P}}_{k\vert k-1} \left(\mathbf{I} - \mathbf{K}_k \mathbf{C}_k\right) ^{\mathrm{T}}  + \mathbf{K}_k \mathbf{R}_k \mathbf{K}_k ^{\mathrm{T}}\)}}                                                                               \\
		                     & = \tilde{\mathbf{P}}_{k\vert k-1} -\mathbf{K}_k \mathbf{C}_k \tilde{\mathbf{P}}_{k\vert k-1} - \tilde{\mathbf{P}}_{k\vert k-1} \mathbf{C}_k ^{\mathrm{T}} \mathbf{K}_k ^{\mathrm{T}} + \mathbf{K}_k \mathbf{C}_k \tilde{\mathbf{P}}_{k\vert k-1} \mathbf{C}_k ^{\mathrm{T}} \mathbf{K}_k ^{\mathrm{T}} + \mathbf{K}_k \mathbf{R}_k \mathbf{K}_k ^{\mathrm{T}} \\
		                     & = \tilde{\mathbf{P}}_{k\vert k-1} -\mathbf{K}_k \mathbf{C}_k \tilde{\mathbf{P}}_{k\vert k-1} - \tilde{\mathbf{P}}_{k\vert k-1} \mathbf{C}_k ^{\mathrm{T}} \mathbf{K}_k ^{\mathrm{T}} + \mathbf{K}_k \mathbf{S}_{k\vert k-1} \mathbf{K}_k ^{\mathrm{T}}
	\end{align}
	\vspace*{\fill}

	\framebreak
	Ready to go,
	\begin{equation}
		\frac{\partial \operatorname{Tr}\left(\tilde{\mathbf{P}}_k\right)}{\partial \mathbf{K}_k} = -2 \tilde{\mathbf{P}}_{k\vert k-1} \mathbf{C}_k ^{\mathrm{T}} + 2 \mathbf{S}_{k\vert k-1} \mathbf{K}_k = \mathbf{0}
	\end{equation}
	Finally,
	\begin{equation}
		\tikzmarknode{optial_kalman_gain}{\colorbox{red_marknode_color}{\(\mathbf{K}_k = \tilde{\mathbf{P}}_{k\vert k-1} \mathbf{C}_k ^{\mathrm{T}} \mathbf{S}_{k\vert k-1}^{-1}\)}}
	\end{equation}
\end{frame}

\section{Summary}
\begin{frame}{Procedure}
	\begin{enumerate}
		\setlength\itemsep{1.5em}
		\item Model the random dynamics of the system
		      \newline
		      \(\mathbf{A}_k, \mathbf{B}_k, \mathbf{C}_k, \mathbf{Q}_k, \mathbf{R}_k \quad \left(\forall k \in \mathbb{N}\right)\)
		\item Estimate the initial state and error covariance
		      \newline
		      \(\hat{\mathbf{x}}_{0\vert 0}, \tilde{\mathbf{P}}_{0\vert 0}\)
		\item Predict
		      \newline
		      \(\hat{\mathbf{x}}_{k\vert k-1 } = \mathbf{A}_k \hat{\mathbf{x}}_{k-1\vert k-1} + \mathbf{B}_{k} \mathbf{u}_{k},\quad \tilde{\mathbf{P}}_{k\vert k-1} = \mathbf{A}_k \mathbf{P}_{k-1\vert k-1} \mathbf{A}_k ^{\mathrm{T}} + \mathbf{R}_k\)
		\item Caculate the Kalman gain
		      \newline
		      \(\mathbf{S}_k = \mathbf{C}_k \tilde{\mathbf{P}}_{k\vert k-1} \mathbf{C}_k ^{\mathrm{T}} + \mathbf{R}_k,\quad \mathbf{K}_k = \hat{\mathbf{P}}_{k\vert k-1} \mathbf{C}_k ^{\mathrm{T}} \mathbf{S}_k^{-1}\)
		\item Correct
		      \newline
		      \(\hat{\mathbf{x}}_{k\vert k} = \left(\mathbf{I}-\mathbf{K}_k \mathbf{C}_k\right)\hat{\mathbf{x}}_{k\vert k-1} + \mathbf{K}_k \mathbf{y}_k, \quad \tilde{\mathbf{P}}_{k\vert k} = \left(\mathbf{I}-\mathbf{K}_k \mathbf{C}_k\right) \tilde{\mathbf{P}}_{k\vert k-1} \left(\mathbf{I}-\mathbf{K}_k \mathbf{C}_k\right)^{\mathrm{T}} + \mathbf{K}_k \mathbf{R}_k \mathbf{K}_k ^{\mathrm{T}} \)
	\end{enumerate}
\end{frame}
\section{Variants}
\subsection{Kalman-Bucy Filter}
\begin{frame}[allowframebreaks]{Underlying dynamic system model}
	\vspace*{\fill}
	\begin{definition}[Underlying dynamic system model of Kalman-Bucy Filter]
		\begin{figure}[htbp]
			\vspace*{2.5em}
			\begin{numcases}{}
				\dot{\mathbf{x}}(\tikzmarknode{time}{\colorbox{blue_marknode_color}{\(t\)}}) = \tikzmarknode{system_matrix}{\colorbox{green_marknode_color}{\(\mathbf{A}\)}}(t) \tikzmarknode{state_vector}{\colorbox{blue_marknode_color}{\(\mathbf{x}\)}}(t) + \tikzmarknode{control_matrix}{\colorbox{green_marknode_color}{\(\mathbf{B}\)}}(t) \tikzmarknode{input_vector}{\colorbox{blue_marknode_color}{\(\mathbf{u}\)}}(t) + \tikzmarknode{process_noise}{\colorbox{red_marknode_color}{\(\mathbf{v}(t)\)}} \\
				\tikzmarknode{output_vector}{\colorbox{blue_marknode_color}{\(\mathbf{y}\)}}(t) = \tikzmarknode{measurement_matrix}{\colorbox{green_marknode_color}{\(\mathbf{C}\)}}(t) \mathbf{x}(t) + \tikzmarknode{measurement_noise}{\colorbox{red_marknode_color}{\(\mathbf{w}(t)\)}}
			\end{numcases}
			\begin{tikzpicture}[overlay,remember picture,>=stealth,nodes={align=left,inner ysep=1pt},<-]
				\path (time.north) ++ (0em,0.5em) node[anchor=south east,color=blue_annotate_color] (time_annotate) {\footnotesize{\(\in \mathbb{R}\), time}};
				\draw [color=blue_annotate_color] (time.north) |- (time_annotate.south west);

				\path (system_matrix.north) ++ (0,2em) node[anchor=south east,color=green_annotate_color] (system_green_annotate) {\footnotesize{\(\in \mathcal{M}(n,n)\), system matrix }};
				\draw [color=green_annotate_color] (system_matrix.north) |- (system_green_annotate.south west);

				\path (state_vector.north) ++ (0em,3.5em) node[anchor=south east,color=blue_annotate_color] (state_blue_annotate) {\footnotesize{\(\in \mathbb{R}^n\), state vector}};
				\draw [color=blue_annotate_color] (state_vector.north) |- (state_blue_annotate.south west);

				\path (control_matrix.north) ++ (0em,2.8em) node[anchor=south west,color=green_annotate_color] (control_green_annotate) {\footnotesize{\(\in \mathcal{M}(n,p)\), control matrix}};
				\draw [color=green_annotate_color] (control_matrix.north) |- (control_green_annotate.south east);

				\path (input_vector.north) ++ (0em,1.5em) node[anchor=south west,color=blue_annotate_color] (input_blue_annotate) {\footnotesize{\(\in \mathbb{R}^p\), control vector}};
				\draw [color=blue_annotate_color] (input_vector.north) |- (input_blue_annotate.south east);

				\path (output_vector.south) ++ (0em,-1.5em) node[anchor=north east,color=blue_annotate_color] (output_blue_annotate) {\footnotesize{\(\in \mathbb{R}^q\), measurement vector}};
				\draw [color=blue_annotate_color] (output_vector.south) |- (output_blue_annotate.south west);

				\path (measurement_matrix.south) ++ (0em,-3.5em) node[anchor=north west,color=green_annotate_color] (measurement_green_annotate) {\footnotesize{\(\in \mathcal{M}(q,n)\), measurement matrix}};
				\draw [color=green_annotate_color] (measurement_matrix.south) |- (measurement_green_annotate.south east);
				\path (process_noise.south) ++ (0em,-0.35em) node[anchor=north west,color=red_annotate_color] (process_noise_annotate) {\footnotesize WGN\footnote{white Gaussian noise} with PSD\footnote{power spectral density}=\(\mathbf{Q}(t)\),\\[-0.4em] \footnotesize process noise};
				\draw [color=red_annotate_color] (process_noise.south) |- (process_noise_annotate.south east);
				\path (measurement_noise.south) ++ (0em,-0.75em) node[anchor=north west,color=red_annotate_color] (measurement_noise_annotate) {\footnotesize WGN with PSD=\(\mathbf{R}(t)\),\\[-0.4em] \footnotesize measurement noise};
				\draw [color=red_annotate_color] (measurement_noise.south) |- (measurement_noise_annotate.south east);
			\end{tikzpicture}
		\end{figure}
		\vspace*{4em}
	\end{definition}
	\vspace*{\fill}

	\framebreak
	\vspace*{\fill}
	\begin{block}{Additional information on \textcolor{red_annotate_color}{white noise}}
		Take the process noise for example,
		\begin{align}
			\operatorname{E}\left(\mathbf{v}(t)\right) = \mathbf{0} \\
			\operatorname{E}\left(\mathbf{v}(t+\tau)\mathbf{v}^{\mathrm{T}}(t)\right) = \mathbf{Q}(t) = \mathbf{Q}\delta(\tau)
		\end{align}
	\end{block}
	\vspace*{\fill}

\end{frame}
% \subsection{Extended Kalman Filter}
% \begin{frame}{Underlying dynamic system model}
% 	\begin{numcases}{}
% 		\dot{\mathbf{x}} = f\left(\mathbf{x}, \mathbf{u}\right) + \mathbf{v} \\
% 		\mathbf{y} = g\left(\mathbf{x}\right) + \mathbf{w}
% 	\end{numcases}
% \end{frame}
% \subsection{Error-state (Indirect) Kalman Filter}
\section{Example application}

\appendix
\section{Appendix}
\begin{frame}[allowframebreaks]{References}
	\printbibliography[heading=none]
\end{frame}
\end{document}
