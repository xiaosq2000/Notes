%! TeX root: ./main.tex

%%%%%%%%%%%%%%%%%%%%%%%%%%%%%%%%%%%%%%%%%%%%%%%%%%%%%%%%%%%%%%%%%%%%%%%%%%%%%%%%
%%%%%%%%%%%%%%%%%%%%%%%%%%%%%%%%%% PREAMABLE %%%%%%%%%%%%%%%%%%%%%%%%%%%%%%%%%%%
%%%%%%%%%%%%%%%%%%%%%%%%%%%%%%%%%%%%%%%%%%%%%%%%%%%%%%%%%%%%%%%%%%%%%%%%%%%%%%%%
\documentclass[utf-8, 10pt, aspectratio=1610]{beamer}
\mode<presentation>
\usetheme[
background=light,
titleformat=regular,
% subsectionpage=progressbar,
subsectionpage=none,
block=fill,
]{metropolis}
\usepackage{appendixnumberbeamer}
\usepackage{fontawesome5}
% fontspec is preloaded
    \setmainfont{Source Serif 4}

\usepackage{xcolor}
    % ref: http://zhongguose.com
    \definecolor{kongquelan}{RGB}{14,176,201}
    \definecolor{shenhuilan}{RGB}{19,44,51}
    \definecolor{jianniaolan}{RGB}{20,145,168}
    \definecolor{koushaolv}{RGB}{93,190,138}
    \definecolor{yingwulv}{RGB}{91,174,35}
    \definecolor{jiguanghong}{RGB}{243,59,31}
    \definecolor{xiangyehong}{RGB}{240,124,130}
    \definecolor{xinghuang}{RGB}{250,142,22}
\usepackage{tikz}
\usetikzlibrary{calc,tikzmark}
\usepackage{tcolorbox}
\colorlet{red_marknode}{xiangyehong!50}
\colorlet{red_annotate}{xiangyehong}
\colorlet{blue_marknode}{jianniaolan!50}
\colorlet{blue_annotate}{jianniaolan}
\colorlet{green_marknode}{yingwulv!50}
\colorlet{green_annotate}{yingwulv}
\usepackage{booktabs}
\usepackage{caption}

\usepackage[backend=biber,natbib=true,style=ext-numeric-comp,sorting=none,backref=true]{biblatex}
\addbibresource{references.bib}
\renewcommand*{\bibfont}{\normalfont\small}
\DeclareOuterCiteDelims{cite}{\textcolor{yingwulv}{\bibopenbracket}}{\textcolor{yingwulv}{\bibclosebracket}}

\usepackage{amsmath}
\usepackage{amsfonts}
\usepackage{amsthm}
\usepackage{mathtools}
\usepackage{cases} % numcases
% ref: https://tex.stackexchange.com/a/602213/240783
\usepackage{empheq}
\usepackage{eqparbox}
\newcommand{\eqmath}[3][c]{%
  % #1 = alignment, default c, #2 = label, #2 = math material
  \eqmakebox[#2][#1]{$\displaystyle#3$}%
}
\newcommand{\eqtext}[3][c]{%
  % #1 = alignment, default c, #2 = label, #2 = text material
  \eqmakebox[#2][#1]{#3}%
}

\hypersetup{
    colorlinks=true,
    linkcolor=xinghuang!80,
    anchorcolor=.,
    filecolor=.,
    menucolor=.,
    runcolor=.,
    urlcolor=jianniaolan,
    citecolor=yingwulv,
}

\title{Measure, Probability and Stochastic Process}
\subtitle{A rigorous but painless introduction}
\author{shuqi}
\date{\today}
\institute{
    \faGithub\;
    \href{https://github.com/xiaosq2000}{xiaosq2000}
    \quad
    \faEnvelope\;
    \href{xiaosq2000@gmail.com}{xiaosq2000@gmail.com}
}

%%%%%%%%%%%%%%%%%%%%%%%%%%%%%%%%%%%%%%%%%%%%%%%%%%%%%%%%%%%%%%%%%%%%%%%%%%%%%%%%
%%%%%%%%%%%%%%%%%%%%%%%%%%%%%%%%%%% DOCUMENT %%%%%%%%%%%%%%%%%%%%%%%%%%%%%%%%%%%
%%%%%%%%%%%%%%%%%%%%%%%%%%%%%%%%%%%%%%%%%%%%%%%%%%%%%%%%%%%%%%%%%%%%%%%%%%%%%%%%
\begin{document}

\begin{frame}
	\titlepage
\end{frame}

% \begin{frame}{Prerequistes \& Gains}
% 	Readers are assumed to have an undergraduate-level basic understanding of
% 	\begin{itemize}
% 		\item Multivariate Probability Theory
% 		\item Linear Algebra
% 		\item Matrix Caculus
% 		\item Fourier Analysis
% 	\end{itemize}
% 	\vspace*{\fill}
% 	\par You are expected to learn diverse related knowledge, such as measure theory, Lebesgue integration, power spectral density, and white noise.
% \end{frame}

\begin{frame}[allowframebreaks]{Outline}
	\setbeamertemplate{section in toc}[sections numbered]
	\tableofcontents[sections={1}]
	\framebreak
	\tableofcontents[sections={2-3}]
	% \tableofcontents[hideallsubsections]
\end{frame}

\section{Measure Theory and Lebesgue Integration}

\subsection{Measure}
\begin{frame}{Measure}
	\vspace*{\fill}
	\begin{block}{Remark\ (motivation of measure)}
		A measure is a generalization and formalization of geometrical measures (length, area, volume) and other common notions, such as magnitude, mass, and probability of events. It is fundamental in many mathematical fields, such as probability and integration.
	\end{block}
	\vspace*{\fill}
	\begin{definition}[measure]
		Let \(X\) be a set and \(\mathcal{F}\) a \(\sigma \text{-algebra} \) over \(X\). A function \(\mu:\mathcal{F}\mapsto \mathbb{R}_{\infty}^{1}\), where \(\mathbb{R}_{\infty}^{1}\) is the extended real number field, is called a measure if the following three conditions hold:
		\begin{enumerate}
			\item empty is zero: \(\mu \left(\emptyset\right) = 0\)
			\item non-negativity: \(\forall\, E \in \mathcal{F} \left( \mu\left(E\right) \geq 0 \right) \)
			\item special countable-additivity: \(\mu\left(\bigcup_{k=1}^{\infty} E_k\right) = \Sigma_{k=1}^{\infty} \mu \left(E_k\right)\), where \(\left\{E_k\right\}^{\infty}_{k=1} \) is all countable collections of pairwise disjoint sets in \(\mathcal{F}\)
		\end{enumerate}
	\end{definition}\label{def:measure}
	\vspace*{\fill}
\end{frame}

\subsection{\(\sigma \text{-algebra} \)}
\begin{frame}{\(\sigma \text{-algebra} \)}
	\begin{block}{Remark\ (\(\sigma \text{-algebra} \))}
		\par The ``\(\sigma \text{-algebra} \)'' and ``countable'' (actually, closely related to \(\sigma \text{-algebra} \)) make the rigorous definition of measure (def.~\ref{def:measure}) peculiar.
		\par You can check out the definition of \(\sigma \text{-algebra}\) and motivations about it in measure theory on Wikipedia \cite{enwiki:1173641705} , which is quite enlightening. In summary,
		\begin{itemize}
			\item Introducing the \href{https://en.wikipedia.org/wiki/Field_of_sets}{set algebra} to deliver the addition-preserving property of a measure is natural, and \(\sigma \text{-algebra} \) is a set algebra with countable-additivity, alias \(\sigma \text{-additivity} \). But why countable? \cite{566154}  is a good explanation.
			\item \href{https://en.wikipedia.org/wiki/Zermelo-Fraenkel_set_theory}{ZFC} (precisely, \href{https://en.wikipedia.org/wiki/Axiom_of_choice}{axiom of choice}) entails \href{https://en.wikipedia.org/wiki/Non-measurable_set}{non-measurable set} of \(\mathbb{R}^n\), i.e., it is actually impossible to assign a length to all subsets of \(\mathbb{R}\) in a way that preserves some natural additivity and translation invariance properties.   The \href{https://en.wikipedia.org/wiki/Vitali_set}{Vitali set} and the \href{https://en.wikipedia.org/wiki/Banach-Tarski_paradox}{Banach–Tarski paradox} are famous examples.
		\end{itemize}
	\end{block}
\end{frame}

\subsection{Measurable space and measure space}
\begin{frame}[allowframebreaks]{Measurable space and measure space}
	\vspace*{\fill}
	\begin{definition}[measurable space and measure space]
		\begin{minipage}[t]{0.4\linewidth}
			\begin{figure}[htbp]
				\caption*{measurable space}
				\vspace*{-1.5em}
				\begin{equation}
					\overbrace{\left(\tikzmarknode{set}{\colorbox{red_marknode}{\(X\)}}, \tikzmarknode{sigma-algebra}{\colorbox{green_marknode}{\(\mathcal{F}\)}} \right)}^{\text{tuple}}
				\end{equation}
				\begin{tikzpicture}[overlay,remember picture,>=stealth,nodes={align=left,inner ysep=1pt},<-]
					\path (set.south) ++ (0em,-0.5em) node[anchor=north east,color=red_annotate] (set_annotate) {\footnotesize{a set}};
					\draw [color=red_annotate] (set.south) |- (set_annotate.south west);
					\path (sigma-algebra.south) ++ (0em,-1.5em) node[anchor=north east,color=green_annotate] (sigma-algebra_annotate) {\footnotesize{a \( \sigma\text{-algebra}\) on \(X\)}};
					\draw [color=green_annotate] (sigma-algebra.south) |- (sigma-algebra_annotate.south west);
				\end{tikzpicture}
			\end{figure}
		\end{minipage}
		\hspace*{\fill}
		\begin{minipage}[t]{0.4\linewidth}
			\begin{figure}[htbp]
				\caption*{measure space}
				\vspace*{-1.5em}
				\begin{equation}
					\overbrace{\left(\tikzmarknode{set}{\colorbox{red_marknode}{\(X\)}}, \tikzmarknode{sigma-algebra}{\colorbox{green_marknode}{\(\mathcal{F}\)}}, \tikzmarknode{measure}{\colorbox{blue_marknode}{\(\mu\)}}\right)}^{\text{triple}}
				\end{equation}
				\begin{tikzpicture}[overlay,remember picture,>=stealth,nodes={align=left,inner ysep=1pt},<-]
					\path (set.south) ++ (0em,-0.5em) node[anchor=north east,color=red_annotate] (set_annotate) {\footnotesize{a set}};
					\draw [color=red_annotate] (set.south) |- (set_annotate.south west);
					\path (sigma-algebra.south) ++ (0em,-1.5em) node[anchor=north east,color=green_annotate] (sigma-algebra_annotate) {\footnotesize{a \( \sigma\text{-algebra}\) on \(X\)}};
					\draw [color=green_annotate] (sigma-algebra.south) |- (sigma-algebra_annotate.south west);
					\path (measure.south) ++ (0em,-2.5em) node[anchor=north east,color=blue_annotate] (measure_annotate) {\footnotesize{a measure on \(\left(X,\mathcal{F}\right)\)}};
					\draw [color=blue_annotate] (measure.south) |- (measure_annotate.south west);
				\end{tikzpicture}
			\end{figure}
		\end{minipage}
		\vspace*{2em}
	\end{definition}
	\vspace*{\fill}
\end{frame}
\subsection{Measurable function}
\begin{frame}{Measurable function}
	\vspace*{\fill}
	\begin{definition}[measurable function]
		Let \(\left(X,\Sigma\right)\) and \(\left(Y,T\right)\) be measurable spaces. A function \(f:X\mapsto Y\) is measurable if and only if
		\begin{equation}
			\forall\, E \in T \left( f^{-1}(E) \in \Sigma\right)
		\end{equation}
	\end{definition}
	\vspace*{\fill}
	\begin{corollary}[]
		\begin{equation}
			f \text{ is measurable}  \Leftrightarrow\sigma\left(f\right) \subset \Sigma,
		\end{equation}
		where \(\sigma\left(f\right)\) is the \href{https://en.wikipedia.org/wiki/\%CE\%A3-algebra\#\%CF\%83-algebra\_generated\_by\_a\_function}{\(\sigma \text{-algebra} \) generated by \(f\)}.
	\end{corollary}
	\vspace*{\fill}
\end{frame}

\subsection{Lebesgue Measure}
\begin{frame}[allowframebreaks]{Lebesgue measure}
	\vspace*{\fill}
	\begin{block}{Remark\ (motivation)}
		Lebesgue measure is the formalized way of assigning a volume to subsets of \(\mathbb{R}^n\), and coincides with the standard measure of length \((\mathbb{R})\), area \((\mathbb{R}^2)\), or volume \((\mathbb{R}^3)\).
	\end{block}
	\vspace*{\fill}
	\begin{definition}[length of an interval]
		For any interval \(\mathrm{I}=\left[a,b\right]\) or \(\left(a,b\right) \subset \mathbb{R}\), let \(\mathfrak{l}\left(\mathrm{I}\right) = b - a \) denote its length.
	\end{definition}
	\vspace*{\fill}

	\framebreak
	\vspace*{\fill}
	\begin{definition}[Lebesgue outer measure]
		For any subset \(\mathrm{E} \subseteq \mathbb{R}\), its Lebesgue outer measure \(\lambda^{\ast} : \mathfrak{p}\left(\mathbb{R}\right) \mapsto \left[0,+\infty\right]\) is defined as\footnote{\(\mathfrak{p}\left(\circ\right)\) denotes the power set of \(\circ\).}
		\begin{equation}
			\lambda^\ast \left(\mathrm{E}\right) := \operatorname{inf}\left\{\sum_{k=1}^{\infty} \mathfrak{l}\left(\mathrm{I}_k\right): \left\{\mathrm{I}_k\right\}_{k\in \mathbb{N}} \text{ is a sequence of open intervals with } \mathrm{E} \subset \bigcup_{k=1}^{\infty}  \mathrm{I}_k\right\}
		\end{equation}
	\end{definition}
	\vspace*{\fill}
	\begin{definition}[Carathéodory's criterion]
		\begin{equation}
			\lambda^{\ast} \left(\mathrm{E}\right) = \lambda^{\ast} \left(\mathrm{E} \cap \mathrm{A}\right) + \lambda^{\ast} \left(\mathrm{E} \cap \mathrm{A}^c \right),\quad \forall\, \mathrm{A} \subset \mathbb{R}
		\end{equation}
	\end{definition}
	\vspace*{\fill}

	\framebreak
	\vspace*{\fill}
	\begin{definition}[Lebesgue measure]
		The set of all such \(\mathrm{E}\subseteq \mathbb{R}\) that fulfills the Carathéodory criterion forms a \(\sigma \text{-algebra} \), its Lebesgue measure is defined to be its Lebesgue outer measure,
		\begin{equation}
			\lambda \left(\mathrm{E}\right) = \lambda^\ast \left(\mathrm{E}\right)
		\end{equation}
	\end{definition}
	\vspace*{\fill}
	\begin{block}{Remark\ (motivation of an outer measure and a criterion)}
		There are four intuitive requirements of a Lebesgue measure, but they are incompatible due to \(ZFC\). The purpose of constructing an outer measure on all subsets is to pick out a class of subsets to be called measurable, in such a way (the criterion) to satisfy the countable additivity property. Check out \cite{enwiki:1171717640} for details.
	\end{block}
	\vspace*{\fill}
\end{frame}

\subsection{Lebesgue integral}
\begin{frame}[allowframebreaks]{Lebesgue integral}
	\vspace*{\fill}
	\begin{block}{Remark}
		Like step functions in Riemann-Darboux's approach, the concept of Lebesgue integration is built on indicator functions.
	\end{block}
	\vspace*{\fill}
	\begin{definition}[indicator function]
		A indicator function of a subset \(S\) of a set \(X\) is denoted by \(\operatorname{1}_S: X \mapsto \left\{0,1\right\}\), defined as
		\begin{empheq}[left={\eqmath[r]{A}{\operatorname{1}_S(x):=}\empheqlbrace}]{alignat=2}
			&\eqmath[l]{B}{1,} & \eqtext[l]{C}{\(x\in S\)} \\
			&\eqmath[l]{B}{0,} & \eqtext[l]{C}{\(x\notin S\)}
		\end{empheq}
	\end{definition}
	\vspace*{\fill}
	\framebreak
	\vspace*{\fill}
	\begin{definition}[Lebesgue integral of an indicator function]
		If the subset \(S\) is measurable consistent with a given measure \(\mu\), the Lebesgue integral of the indicator function is defined as
		\begin{equation}
			\int_{X}^{} \operatorname{1}_S \operatorname{d}\!\mu := \int_{S}^{} \operatorname{1}_S \operatorname{d}\!\mu := \mu \left(S\right)
		\end{equation}
		Note that the integral may be equal to \(\infty\), unless \(\mu\) is a finite measure.
		\par If \(T\) is another measurable subset of \(X\),
		\begin{equation}
			\int_{T}^{} \operatorname{1}_S \operatorname{d}\!\mu := \mu \left(S \cap T\right)
		\end{equation}
	\end{definition}
	\vspace*{\fill}
	\framebreak
	\vspace*{\fill}
	\begin{definition}[simple function]
		A simple function is a finite linear combination of indicator functions
		\begin{equation}
			\Sigma_{k=0}^N a_k \operatorname{1}_{S_k}
		\end{equation}
		If \(a_k\in \mathbb{R}\) and \(S_k\) are disjoint measureable sets, it is called a measurable simple function.
	\end{definition}
	\vspace*{\fill}
	\begin{definition}[Lebesgue integral of a measurable simple function]
		Linearity is a desired property for Lebesgue integrals, i.e.,its Lebesgue outer measure is defined as
		\vspace*{-1.5em}
		\begin{figure}[htbp]
			\begin{equation}
				\int_{X}^{} \Sigma_{k=1}^N a_k \operatorname{1}_{S_k} \operatorname{d}\! \mu := \Sigma_{k=1}^N a_k \int_{X}^{} \operatorname{1}_{S_k} \operatorname{d}\! \mu = \Sigma_{k=1}^N a_k \mu \left(S_k\right), \quad \text{if } \tikzmarknode{condition}{\colorbox{green_marknode}{\(\left(\forall\,k : \mu_k = \infty \Rightarrow a_k = 0 \right)\)}}
			\end{equation}
			\begin{tikzpicture}[overlay,remember picture,>=stealth,nodes={align=left,inner ysep=1pt},<-]
				\path (condition.south) ++ (0em,-1em) node[anchor=north west,color=green_annotate] (condition_annotate) {\footnotesize{to avoid \(\infty-\infty\)}};
				\draw [color=green_annotate] (condition.south) |- (condition_annotate.south east);
			\end{tikzpicture}
		\end{figure}
	\end{definition}
	\vspace*{\fill}
	\framebreak
	\vspace*{\fill}
	\begin{quote}
		To compute the Riemann integral of \(f\), one partitions the domain \([a, b]\) into subintervals; while in the Lebesgue integral, one is in effect partitioning the range of \(f\).
		\flushright ---\;\textrm{Folland}
	\end{quote}
	\vspace*{\fill}
	\begin{definition}[Lebesgue integral of a non-negative measurable function]
		\begin{equation}
			\int_{E}^{} f \operatorname{d}\! \mu := \operatorname{sup}\left\{ \int_{E}^{} s \operatorname{d}\! \mu: 0<s<f\,\wedge\,s \text{ is simple} \right\}
		\end{equation}
	\end{definition}
	\vspace*{\fill}
	\framebreak
	\vspace*{\fill}
	\begin{definition}[Lebesgue integral of a measurable function]
		\begin{equation}
			f = f^{\,+} + f^{\,-}
		\end{equation}, where
		\par \nointerlineskip
		\begin{minipage}[t]{0.47\linewidth}
			\begin{empheq}[left={\eqmath[r]{A}{f^{\,+} := }\empheqlbrace}]{alignat=2}
				&\eqmath[l]{B}{f(x)} ,\quad& \eqtext[l]{C}{\(f(x)>0\)} \\
				&\eqmath[l]{B}{0} ,\quad& \eqtext[l]{C}{otherwise}
			\end{empheq}
		\end{minipage}
		\hspace*{\fill}
		\begin{minipage}[t]{0.47\linewidth}
			\begin{empheq}[left={\eqmath[r]{A}{f^{\,-} :=}\empheqlbrace}]{alignat=2}
				&\eqmath[l]{B}{-f(x)} ,\quad & \eqtext[l]{C}{\(f(x)<0\)} \\
				&\eqmath[l]{B}{0} ,\quad & \eqtext[l]{C}{otherwise}
			\end{empheq}
		\end{minipage}
		\par Then,
		\vspace*{-2\baselineskip}
		\begin{figure}[htbp]
			\begin{equation}
				\int_{}^{} f \operatorname{d}\! \mu := \int_{}^{} f^{\,+} \operatorname{d}\! \mu - \int_{}^{} f^{\,-} \operatorname{d}\! \mu, \quad \text{if } \tikzmarknode{at_least_one_finite}{\colorbox{green_marknode}{\(\displaystyle \operatorname{min} \left(\int_{}^{} f^+ \operatorname{d}\! \mu , \int_{}^{} f^- \operatorname{d}\! \mu \right) < \infty\)}}
			\end{equation}
			\begin{tikzpicture}[overlay,remember picture,>=stealth,nodes={align=left,inner ysep=1pt},<-]
				\path (at_least_one_finite.south) ++ (0em,-1em) node[anchor=north east,color=green_annotate] (at_least_one_finite_annotate) {\footnotesize{at least one is finite, to avoid \(\infty-\infty\)}};
				\draw [color=green_annotate] (at_least_one_finite.south) |- (at_least_one_finite_annotate.south west);
			\end{tikzpicture}
		\end{figure}
	\end{definition}
\end{frame}

\subsection{Radon–Nikodym theorem}
\begin{frame}[allowframebreaks]{Radon–Nikodym theorem}
	\vspace*{\fill}
	\begin{definition}[absolute continuity of measures]

	\end{definition}
	\vspace*{\fill}
	\framebreak
	\vspace*{\fill}
	\begin{definition}[Radon–Nikodym theorem]

	\end{definition}
	\vspace*{\fill}
\end{frame}


\section{Probability Theory}
\subsection{Probability}
\begin{frame}{Probability}
	\begin{definition}[probability]
		\begin{figure}[htbp]
			\caption*{Kolmogorov Axioms}
			\vspace*{-1.5em}
			\begin{equation}
				\overbrace{\left(\tikzmarknode{sample_space}{\colorbox{red_marknode}{\(\Omega\)}},\tikzmarknode{event_space}{\colorbox{green_marknode}{\(\mathcal{F}\)}},\tikzmarknode{probability_measure}{\colorbox{blue_marknode}{\(P\)}}\right)}^{\text{probability space}}
			\end{equation}
			\begin{tikzpicture}[overlay,remember picture,>=stealth,nodes={align=left,inner ysep=1pt},<-]
				\path (sample_space.south) ++ (0em,-0.5em) node[anchor=north east,color=red_annotate] (sample_space_annotate) {\footnotesize{sample space}};
				\draw [color=red_annotate] (sample_space.south) |- (sample_space_annotate.south west);
				\path (event_space.south) ++ (0em,-1.75em) node[anchor=north east,color=green_annotate] (event_space_annotate) {\footnotesize{event space}};
				\draw [color=green_annotate] (event_space.south) |- (event_space_annotate.south west);
				\path (probability_measure.south) ++ (0em,-0.5em) node[anchor=north west,color=blue_annotate] (probability_measure_annotate) {\footnotesize{probability measure}};
				\draw [color=blue_annotate] (probability_measure.south) |- (probability_measure_annotate.south east);
			\end{tikzpicture}
		\end{figure}
		the probability is a measure with two additional properties,

		\begin{enumerate}
			\item finiteness: \(\forall E\in \mathcal{F} \left( P(E) \in \mathbb{R}\right)\)
			\item unitarity: \(P(\Omega) = 1\)
		\end{enumerate}

	\end{definition}
\end{frame}
\subsection{Random variable}
\begin{frame}{Random variable}
	\begin{definition}[random variable]
		\begin{figure}[htbp]
			\begin{equation}
				\tikzmarknode{random_variable}{\colorbox{blue_marknode}{\(X\)}}:\tikzmarknode{sample_space}{\colorbox{red_marknode}{\(\Omega\)}} \mapsto \tikzmarknode{state_space}{\colorbox{green_marknode}{\(S\)}}
			\end{equation}
			\begin{tikzpicture}[overlay,remember picture,>=stealth,nodes={align=left,inner ysep=1pt},<-]
				\path (random_variable.north) ++ (0em,+1em) node[anchor=south east,color=blue_annotate] (random_variable_annotate) {\footnotesize{measurable function}};
				\draw [color=blue_annotate] (random_variable.north) |- (random_variable_annotate.south west);
				\path (sample_space.south) ++ (0em,-0.5em) node[anchor=north east,color=red_annotate] (sample_space_annotate) {\footnotesize{sample space}};
				\draw [color=red_annotate] (sample_space.south) |- (sample_space_annotate.south west);
				\path (state_space.south) ++ (0em,-0.5em) node[anchor=north west,color=green_annotate] (state_space_annotate) {\footnotesize{state space}};
				\draw [color=green_annotate] (state_space.south) |- (state_space_annotate.south east);
			\end{tikzpicture}
		\end{figure}
	\end{definition}

	\vspace*{\fill}

	\begin{block}{Remark\ (random variable)}
		\par It is a function but called "variable" to emphsaize on its codomain(state space), usually subsets of \(\mathbb{R}^n\) or \(\mathbb{Z}^n\), which is more convenient for manipulation than the abstract sample space. For example, the event \(E:= \left\{\omega \in \Omega: u < X(\omega) \leq v\right\}\) is usually denoted by \(u < X \leq v\), since \(\omega \in X^{-1}\left(\left(u,v\right]\right) \Leftrightarrow u < X \left( \omega\right) \leq v  \).
	\end{block}
\end{frame}

\section{Stochastic Process}
\subsection{Stochastic process}
\begin{frame}{Stochastic process}
	\vspace*{\fill}
	\begin{definition}[stochastic process]
		A stochastic process is collection of indexed random variables, denoted by
		\begin{equation}
			\left\{X(t): t\in T\right\},
		\end{equation}
		where \(T\) is the index/parameter set.
	\end{definition}
	\vspace*{\fill}
	\begin{block}{Remark\ (index set)}
		\(t \) usually has a physical meaning of time(continuous) or timestamp(discrete).
	\end{block}
	\vspace*{\fill}
\end{frame}

\subsection{Autocorrelation and autocovariance}
\begin{frame}{Autocorrelation and autocovariance}
	\vspace*{\fill}
	Let \(\mathbf{x}(\omega, t): \Omega\times \mathbb{R}\mapsto \mathbb{R}^n\) be a continuous-time multivariate real-valued stochastic process\footnote{The continuous-time multivariate real-valued stochastic process is the most common in the author's background, so if not mentioned, the following definitions, remarks, etc., are based on it.},
	\begin{definition}[autocorrelation]
		\begin{equation}
			\mathbf{R}_{\mathbf{x}\mathbf{x}}(t_1,t_2) = \operatorname{E}\left(\mathbf{x}(t_1) \mathbf{x}(t_2)^{\mathrm{T}}\right)
		\end{equation}
	\end{definition}
	\vspace*{\fill}
	\begin{definition}[autocovariance]
		\begin{equation}
			\mathbf{K}_{\mathbf{x}\mathbf{x}}(t_1,t_2) = \operatorname{Cov}\left(\mathbf{x}(t_1), \mathbf{x}(t_2)\right) = \operatorname{E}\left(\left(\mathbf{x}(t_1) - \operatorname{E}\left(\mathbf{x}(t_1)\right)\right)\left(\mathbf{x}(t_2) - \operatorname{E}\left(\mathbf{x}(t_2)\right)\right)^{\mathrm{T}}\right)
		\end{equation}
	\end{definition}
	\vspace*{\fill}
\end{frame}

\subsection{Stationarity}
\begin{frame}[allowframebreaks]{Stationarity}
	\vspace*{\fill}
	\begin{definition}[strict stationary process]
		\par Let \(F_X\left(X_{t_1+\tau},\cdots,X_{t_n+\tau}\right)\) be the cumulative distribution function(CDF) of the unconditional joint distribution of the stochastic process \(\left\{X_t\right\}\) at times \(t_1+\tau,\cdots,t_n+\tau\).
		\par \(\left\{X_t\right\}\) is a (strict(ly)/strong(ly)) stationary process, if the unconditional joint CDF does not change when shifted in time, i.e.
		\begin{equation}
			\left(\forall\, \tau,t_1,\cdots,t_n\in \mathbb{R}\right)\left(\forall\,n\in \mathbb{N}_{+}\right)\left(F_X\left(x_{t_1+\tau},\cdots,x_{t_n+\tau}\right)\right)
		\end{equation}\label{eq:constraint-of-stationary-process}
	\end{definition}
	\vspace*{\fill}

	\framebreak
	\vspace*{\fill}
	\begin{definition}[wide stationary process]
		A wide/weak stationary process loosens the constraints on CDF(eq.~\ref{eq:constraint-of-stationary-process}) to the following first two conditions, with an additional ``finite second-moment'' condition (eq.~\ref{eq:finite-second-moment}).
		\begin{align}
			 & \operatorname{E}\left(\mathbf{x}(t+\tau)\right) = \operatorname{E}\left(\mathbf{x}(t)\right),                                  & \forall\, t,\tau \in \mathbb{R}                            \\
			 & \operatorname{K}_{\mathbf{x}\mathbf{x}}\left(t_1, t_2\right) = \operatorname{K}_{\mathbf{x}\mathbf{x}}\left(t_1-t_2, 0\right), & \forall\, t_1,t_2 \in \mathbb{R}                           \\
			 & \operatorname{E}\left(\vert \mathbf{x}_t \vert^2 \right) < \infty,                                                             & \forall\, t \in \mathbb{R} \label{eq:finite-second-moment}
		\end{align}
	\end{definition}
	\vspace*{\fill}

	\framebreak
	\vspace*{\fill}
	\begin{corollary}[wide stationary process]
		\begin{itemize}
			\item The expectation is always a constant.
			\item The autocovariance and autocorrelation are better indexed by one variable (time difference) instead of two (timestamps).
			\item Any strictly stationary process which has a finite mean and a covariance is also a wide-sense stationary process.
		\end{itemize}
	\end{corollary}
	\vspace*{\fill}

	\framebreak
	\vspace*{\fill}
	\begin{block}{Remark\ (motivation of wide-sense stationarity, WSS)}
		\par The ``finite second-moment'' condition(eq.~\ref{eq:finite-second-moment}) may remind you of the \href{https://en.wikipedia.org/wiki/Hilbert_space}{Hilbert space}, if you are good at associating.
		\par The Wikipedia \cite{enwiki:1166251052} has a wonderful explanation of its mathematical motivation and the reason why the WSS assumption is widely employed in signal processing algorithms.
	\end{block}
	\vspace*{\fill}
\end{frame}

\subsection{Spectral analysis}
\begin{frame}[allowframebreaks]{Spectral analysis}
	\begin{definition}[energy]
		\begin{equation}
			E := \int_{-\infty}^{+\infty} \Vert \mathbf{x}(t) \Vert^2 \operatorname{d}\! t
		\end{equation}
	\end{definition}
	\begin{theorem}[Parseval's theorem]
		\begin{equation}
			\int_{-\infty}^{+\infty} \Vert \mathbf{x}(t) \Vert^2 \operatorname{d}\! t = \int_{-\infty}^{+\infty} \Vert \hat{\mathbf{x}}(f) \Vert^2 \operatorname{d}\! f,
		\end{equation}
		where \(\hat{\mathbf{x}}(f)\) is the Fourier transform of \(\mathbf{x}(t)\), i.e.,
		\begin{equation}
			\hat{\mathbf{x}}(f) = \int_{-\infty}^{+\infty} \mathrm{e}^{-\mathrm{i}2\pi f t} \mathbf{x}(t) \operatorname{d}\! t
		\end{equation}
	\end{theorem}

	\framebreak
	\begin{definition}[energy spectral density]
		\begin{equation}
			\bar{S}_{\mathbf{xx}} := \Vert \hat{\mathbf{x}}(f) \Vert^2
		\end{equation}
	\end{definition}
\end{frame}

\appendix
\section{References}
\begin{frame}[allowframebreaks]{References}
	\printbibliography[heading=none]
\end{frame}
\end{document}
