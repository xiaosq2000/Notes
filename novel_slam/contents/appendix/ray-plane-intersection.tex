\mode<article>{\section{Ray-plane Intersection}\label{appendix:ray-plane-intersection}}
\begin{frame}<all>[c]
	\mode<presentation>{\Frametitle{Ray-plane Intersection}}
	\par Let the ray and the plane be given parametrically by equation~\ref{eq:appendix-ray} and equation~\ref{eq:appendix-plane}.
	\par
	\begin{minipage}[t]{0.45\linewidth}
		\resetcolorseries[4]{marknode-color-series}
		\resetcolorseries[4]{annotation-color-series}
		\begin{equation}
			\label{eq:appendix-ray}
			\tikzmarknode{n0}{\colorbox{marknode-color-series!![0]}{\(\mathbf{q}\)}}
			=
			\tikzmarknode{n1}{\colorbox{marknode-color-series!![1]}{\(\mathbf{p}\)}}
			+
			\tikzmarknode{n2}{\colorbox{marknode-color-series!![2]}{\(t\)}}
			\tikzmarknode{n3}{\colorbox{marknode-color-series!![3]}{\(\mathbf{v}\)}}
			,
		\end{equation}
		\vspace*{5em}
		\begin{annotatedEquationEnv}
			\annotatedEquation{colorseries}{n0}{south}{0em}{-0.5em}{north east}{annotation-color-series}{\(\in \mathbb{R}^3\), point on the ray}{west}
			\annotatedEquation{colorseries}{n1}{south}{0em}{-4em}{north west}{annotation-color-series}{\(\in \mathbb{R}^3\), initial point}{east}
			\annotatedEquation{colorseries}{n2}{south}{0em}{-3em}{north west}{annotation-color-series}{\(\in \mathbb{R}, t \geq 0\), length}{east}
			\annotatedEquation{colorseries}{n3}{south}{0em}{-0.5em}{north west}{annotation-color-series}{\(\in \mathbb{R}^3, \Vert \mathbf{v} \Vert = 1\), \\direction vector}{east}
		\end{annotatedEquationEnv}
	\end{minipage}
	\begin{minipage}[t]{0.45\linewidth}
		\resetcolorseries[3]{marknode-color-series}
		\resetcolorseries[3]{annotation-color-series}
		\begin{equation}
			\label{eq:appendix-plane}
			\langle
			\tikzmarknode{n0}{\colorbox{marknode-color-series!![0]}{\(\mathbf{n}\)}}
			,
			\tikzmarknode{n1}{\colorbox{marknode-color-series!![1]}{\(\mathbf{r}\)}}
			\rangle
			+
			\tikzmarknode{n2}{\colorbox{marknode-color-series!![2]}{\(d\)}}
			= 0.
		\end{equation}
		\begin{annotatedEquationEnv}
			\annotatedEquation{colorseries}{n0}{south}{0em}{-3.5em}{north west}{annotation-color-series}{\(\in \mathbb{R}^3\), normal vector}{east}
			\annotatedEquation{colorseries}{n1}{south}{0em}{-2em}{north west}{annotation-color-series}{\(\in \mathbb{R}^3\), point on the plane}{east}
			\annotatedEquation{colorseries}{n2}{south}{0em}{-0.5em}{north west}{annotation-color-series}{\(\in \mathbb{R}\), translation}{east}
		\end{annotatedEquationEnv}
	\end{minipage}
	\par\noindent The ray-plane intersection occurs when \(\mathbf{q}\) satisfies the plane equation. Substituting and rearranging, we can acquire the length from the initial point of the ray and the intersection,
	\begin{align}
		% \label{eq:}
		0 & = \langle \mathbf{n}, \left( \mathbf{p} + t \mathbf{v} \right) \rangle + d                    \\
		0 & = \langle \mathbf{n}, \mathbf{p} \rangle + t \langle \mathbf{n},\mathbf{v} \rangle + d        \\
		t & = -\frac{d + \langle \mathbf{n}, \mathbf{p} \rangle}{\langle \mathbf{n}, \mathbf{v} \rangle}.
	\end{align}
\end{frame}
